% Options for packages loaded elsewhere
% Options for packages loaded elsewhere
\PassOptionsToPackage{unicode}{hyperref}
\PassOptionsToPackage{hyphens}{url}
\PassOptionsToPackage{dvipsnames,svgnames,x11names}{xcolor}
%
\documentclass[
  10pt,
  a4paper,
]{article}
\usepackage{xcolor}
\usepackage[left=2cm,right=2cm,top=2.5cm,bottom=2.5cm]{geometry}
\usepackage{amsmath,amssymb}
\setcounter{secnumdepth}{-\maxdimen} % remove section numbering
\usepackage{iftex}
\ifPDFTeX
  \usepackage[T1]{fontenc}
  \usepackage[utf8]{inputenc}
  \usepackage{textcomp} % provide euro and other symbols
\else % if luatex or xetex
  \usepackage{unicode-math} % this also loads fontspec
  \defaultfontfeatures{Scale=MatchLowercase}
  \defaultfontfeatures[\rmfamily]{Ligatures=TeX,Scale=1}
\fi
\usepackage{lmodern}
\ifPDFTeX\else
  % xetex/luatex font selection
\fi
% Use upquote if available, for straight quotes in verbatim environments
\IfFileExists{upquote.sty}{\usepackage{upquote}}{}
\IfFileExists{microtype.sty}{% use microtype if available
  \usepackage[]{microtype}
  \UseMicrotypeSet[protrusion]{basicmath} % disable protrusion for tt fonts
}{}
\makeatletter
\@ifundefined{KOMAClassName}{% if non-KOMA class
  \IfFileExists{parskip.sty}{%
    \usepackage{parskip}
  }{% else
    \setlength{\parindent}{0pt}
    \setlength{\parskip}{6pt plus 2pt minus 1pt}}
}{% if KOMA class
  \KOMAoptions{parskip=half}}
\makeatother
% Make \paragraph and \subparagraph free-standing
\makeatletter
\ifx\paragraph\undefined\else
  \let\oldparagraph\paragraph
  \renewcommand{\paragraph}{
    \@ifstar
      \xxxParagraphStar
      \xxxParagraphNoStar
  }
  \newcommand{\xxxParagraphStar}[1]{\oldparagraph*{#1}\mbox{}}
  \newcommand{\xxxParagraphNoStar}[1]{\oldparagraph{#1}\mbox{}}
\fi
\ifx\subparagraph\undefined\else
  \let\oldsubparagraph\subparagraph
  \renewcommand{\subparagraph}{
    \@ifstar
      \xxxSubParagraphStar
      \xxxSubParagraphNoStar
  }
  \newcommand{\xxxSubParagraphStar}[1]{\oldsubparagraph*{#1}\mbox{}}
  \newcommand{\xxxSubParagraphNoStar}[1]{\oldsubparagraph{#1}\mbox{}}
\fi
\makeatother

\usepackage{color}
\usepackage{fancyvrb}
\newcommand{\VerbBar}{|}
\newcommand{\VERB}{\Verb[commandchars=\\\{\}]}
\DefineVerbatimEnvironment{Highlighting}{Verbatim}{commandchars=\\\{\}}
% Add ',fontsize=\small' for more characters per line
\usepackage{framed}
\definecolor{shadecolor}{RGB}{241,243,245}
\newenvironment{Shaded}{\begin{snugshade}}{\end{snugshade}}
\newcommand{\AlertTok}[1]{\textcolor[rgb]{0.68,0.00,0.00}{#1}}
\newcommand{\AnnotationTok}[1]{\textcolor[rgb]{0.37,0.37,0.37}{#1}}
\newcommand{\AttributeTok}[1]{\textcolor[rgb]{0.40,0.45,0.13}{#1}}
\newcommand{\BaseNTok}[1]{\textcolor[rgb]{0.68,0.00,0.00}{#1}}
\newcommand{\BuiltInTok}[1]{\textcolor[rgb]{0.00,0.23,0.31}{#1}}
\newcommand{\CharTok}[1]{\textcolor[rgb]{0.13,0.47,0.30}{#1}}
\newcommand{\CommentTok}[1]{\textcolor[rgb]{0.37,0.37,0.37}{#1}}
\newcommand{\CommentVarTok}[1]{\textcolor[rgb]{0.37,0.37,0.37}{\textit{#1}}}
\newcommand{\ConstantTok}[1]{\textcolor[rgb]{0.56,0.35,0.01}{#1}}
\newcommand{\ControlFlowTok}[1]{\textcolor[rgb]{0.00,0.23,0.31}{\textbf{#1}}}
\newcommand{\DataTypeTok}[1]{\textcolor[rgb]{0.68,0.00,0.00}{#1}}
\newcommand{\DecValTok}[1]{\textcolor[rgb]{0.68,0.00,0.00}{#1}}
\newcommand{\DocumentationTok}[1]{\textcolor[rgb]{0.37,0.37,0.37}{\textit{#1}}}
\newcommand{\ErrorTok}[1]{\textcolor[rgb]{0.68,0.00,0.00}{#1}}
\newcommand{\ExtensionTok}[1]{\textcolor[rgb]{0.00,0.23,0.31}{#1}}
\newcommand{\FloatTok}[1]{\textcolor[rgb]{0.68,0.00,0.00}{#1}}
\newcommand{\FunctionTok}[1]{\textcolor[rgb]{0.28,0.35,0.67}{#1}}
\newcommand{\ImportTok}[1]{\textcolor[rgb]{0.00,0.46,0.62}{#1}}
\newcommand{\InformationTok}[1]{\textcolor[rgb]{0.37,0.37,0.37}{#1}}
\newcommand{\KeywordTok}[1]{\textcolor[rgb]{0.00,0.23,0.31}{\textbf{#1}}}
\newcommand{\NormalTok}[1]{\textcolor[rgb]{0.00,0.23,0.31}{#1}}
\newcommand{\OperatorTok}[1]{\textcolor[rgb]{0.37,0.37,0.37}{#1}}
\newcommand{\OtherTok}[1]{\textcolor[rgb]{0.00,0.23,0.31}{#1}}
\newcommand{\PreprocessorTok}[1]{\textcolor[rgb]{0.68,0.00,0.00}{#1}}
\newcommand{\RegionMarkerTok}[1]{\textcolor[rgb]{0.00,0.23,0.31}{#1}}
\newcommand{\SpecialCharTok}[1]{\textcolor[rgb]{0.37,0.37,0.37}{#1}}
\newcommand{\SpecialStringTok}[1]{\textcolor[rgb]{0.13,0.47,0.30}{#1}}
\newcommand{\StringTok}[1]{\textcolor[rgb]{0.13,0.47,0.30}{#1}}
\newcommand{\VariableTok}[1]{\textcolor[rgb]{0.07,0.07,0.07}{#1}}
\newcommand{\VerbatimStringTok}[1]{\textcolor[rgb]{0.13,0.47,0.30}{#1}}
\newcommand{\WarningTok}[1]{\textcolor[rgb]{0.37,0.37,0.37}{\textit{#1}}}

\usepackage{longtable,booktabs,array}
\usepackage{calc} % for calculating minipage widths
% Correct order of tables after \paragraph or \subparagraph
\usepackage{etoolbox}
\makeatletter
\patchcmd\longtable{\par}{\if@noskipsec\mbox{}\fi\par}{}{}
\makeatother
% Allow footnotes in longtable head/foot
\IfFileExists{footnotehyper.sty}{\usepackage{footnotehyper}}{\usepackage{footnote}}
\makesavenoteenv{longtable}
\usepackage{graphicx}
\makeatletter
\newsavebox\pandoc@box
\newcommand*\pandocbounded[1]{% scales image to fit in text height/width
  \sbox\pandoc@box{#1}%
  \Gscale@div\@tempa{\textheight}{\dimexpr\ht\pandoc@box+\dp\pandoc@box\relax}%
  \Gscale@div\@tempb{\linewidth}{\wd\pandoc@box}%
  \ifdim\@tempb\p@<\@tempa\p@\let\@tempa\@tempb\fi% select the smaller of both
  \ifdim\@tempa\p@<\p@\scalebox{\@tempa}{\usebox\pandoc@box}%
  \else\usebox{\pandoc@box}%
  \fi%
}
% Set default figure placement to htbp
\def\fps@figure{htbp}
\makeatother





\setlength{\emergencystretch}{3em} % prevent overfull lines

\providecommand{\tightlist}{%
  \setlength{\itemsep}{0pt}\setlength{\parskip}{0pt}}



 


\usepackage{booktabs}
\usepackage{longtable}
\usepackage{array}
\usepackage{multirow}
\usepackage{wrapfig}
\usepackage{float}
\usepackage{colortbl}
\usepackage{pdflscape}
\usepackage{tabu}
\usepackage{threeparttable}
\usepackage{threeparttablex}
\usepackage[normalem]{ulem}
\usepackage{makecell}
\usepackage{xcolor}
\makeatletter
\@ifpackageloaded{caption}{}{\usepackage{caption}}
\AtBeginDocument{%
\ifdefined\contentsname
  \renewcommand*\contentsname{Table of contents}
\else
  \newcommand\contentsname{Table of contents}
\fi
\ifdefined\listfigurename
  \renewcommand*\listfigurename{List of Figures}
\else
  \newcommand\listfigurename{List of Figures}
\fi
\ifdefined\listtablename
  \renewcommand*\listtablename{List of Tables}
\else
  \newcommand\listtablename{List of Tables}
\fi
\ifdefined\figurename
  \renewcommand*\figurename{Figure}
\else
  \newcommand\figurename{Figure}
\fi
\ifdefined\tablename
  \renewcommand*\tablename{Table}
\else
  \newcommand\tablename{Table}
\fi
}
\@ifpackageloaded{float}{}{\usepackage{float}}
\floatstyle{ruled}
\@ifundefined{c@chapter}{\newfloat{codelisting}{h}{lop}}{\newfloat{codelisting}{h}{lop}[chapter]}
\floatname{codelisting}{Listing}
\newcommand*\listoflistings{\listof{codelisting}{List of Listings}}
\makeatother
\makeatletter
\makeatother
\makeatletter
\@ifpackageloaded{caption}{}{\usepackage{caption}}
\@ifpackageloaded{subcaption}{}{\usepackage{subcaption}}
\makeatother
\usepackage{bookmark}
\IfFileExists{xurl.sty}{\usepackage{xurl}}{} % add URL line breaks if available
\urlstyle{same}
\hypersetup{
  pdftitle={Advanced Techniques for Multiperiod Multiobjective Portfolio Optimization in Agricultural Commodity Markets: Integrating GAMLSS, Markov-Switching GARCH, and Reinforcement Learning},
  pdfauthor={Rodrigo Hermont Ozon; Gilberto Reynoso-Meza},
  pdfkeywords={Agricultural commodities, Portfolio
optimization, GAMLSS, Markov-switching GARCH, Multi-objective
optimization, NSGA-II, Reinforcement learning, Risk
management, Volatility forecasting, Brazilian markets},
  colorlinks=true,
  linkcolor={blue},
  filecolor={Maroon},
  citecolor={Blue},
  urlcolor={Blue},
  pdfcreator={LaTeX via pandoc}}


\title{Advanced Techniques for Multiperiod Multiobjective Portfolio
Optimization in Agricultural Commodity Markets: Integrating GAMLSS,
Markov-Switching GARCH, and Reinforcement Learning}
\author{Rodrigo Hermont Ozon \and Gilberto Reynoso-Meza}
\date{2025-10-22}
\begin{document}
\maketitle
\begin{abstract}
\textbf{Background}: Agricultural commodity markets exhibit persistent
volatility regime shifts, heavy-tailed return distributions, and
nonlinear price dynamics that challenge traditional portfolio
optimization approaches. These characteristics require sophisticated
modeling frameworks that can simultaneously capture distributional
complexities, regime-dependent volatility, and dynamic decision-making
under uncertainty.

\textbf{Objective}: This research develops an integrated methodological
framework combining Generalized Additive Models for Location, Scale, and
Shape (GAMLSS), Markov-Switching GARCH (MSGARCH), multi-objective
optimization via evolutionary algorithms (NSGA-II, Differential
Evolution), and Reinforcement Learning (RL) for dynamic asset allocation
in agricultural commodity portfolios.

\textbf{Methods}: We apply a four-stage analytical pipeline: (i)
distributional characterization using GAMLSS to model non-normal return
features; (ii) regime-aware volatility forecasting through MSGARCH
models capturing transitions between calm and turbulent market states;
(iii) multi-objective portfolio optimization balancing return, risk, and
diversification objectives; and (iv) reinforcement learning-based
dynamic allocation that adapts to changing market conditions. The
framework is validated using Brazilian agricultural commodity data
(corn, soybeans, wheat, coffee) spanning 2014-2024.

\textbf{Results}: Empirical analysis demonstrates that GAMLSS
successfully captures negative skewness and excess kurtosis in commodity
returns. MSGARCH models identify two distinct volatility regimes with
high persistence (transition probabilities \textgreater0.90).
Multi-objective optimization generates diverse Pareto-efficient
portfolios with Sharpe ratios ranging from 0.15 to 0.45. The
reinforcement learning agent achieves superior risk-adjusted returns
with 18\% lower maximum drawdowns compared to static equal-weight and
mean-variance strategies.

\textbf{Conclusions}: The integrated framework offers substantial
improvements over traditional approaches across multiple performance
dimensions. Practical applications include enhanced risk management
through better tail risk capture, adaptive allocation strategies
responding to regime shifts, and multi-period optimization accounting
for transaction costs and rebalancing constraints. This methodology
provides actionable insights for risk managers and institutional
investors operating in agricultural commodity markets.
\end{abstract}


\section{Introduction}\label{sec-introduction}

Agricultural commodity markets play a critical role in global food
security, economic stability, and inflation dynamics
{[}\textbf{Gilbert2010?},\textbf{Rossi2013?}{]}. These markets are
characterized by high price volatility, abrupt regime transitions
between periods of calm and turbulence, and significant exposure to
external shocks including weather extremes, geopolitical tensions,
supply chain disruptions, and macroeconomic policy changes
{[}\textbf{FAO2025?},\textbf{WorldBank2025?}{]}. The COVID-19 pandemic
and the Russia-Ukraine conflict have further highlighted the
vulnerability of agricultural supply chains and the importance of
sophisticated risk management strategies {[}\textbf{Ozdemir2025?}{]}.

Traditional portfolio optimization approaches, which typically assume
normally distributed returns and constant volatility, have proven
inadequate for capturing the complex dynamics of agricultural commodity
markets {[}\textbf{Markowitz1952?},\textbf{Engle2004?}{]}. Recent
advances in econometric modeling, multi-objective optimization, and
reinforcement learning offer promising avenues for addressing these
limitations, yet a comprehensive integration of these methodologies
specifically tailored to agricultural commodities remains underdeveloped
in the literature.

\subsection{Research Problem and Motivation}\label{sec-problem}

The central research problem addressed in this study is the development
of a comprehensive methodological framework that simultaneously handles
three interrelated challenges in agricultural commodity portfolio
management:

\textbf{First}, \textbf{distributional inadequacy}: Commodity returns
exhibit persistent deviations from normality, including negative
skewness (higher probability of extreme negative returns), excess
kurtosis (heavy tails), and time-varying higher moments
{[}\textbf{Stasinopoulos2007?}{]}. Traditional mean-variance
optimization and standard GARCH models fail to adequately capture these
features, potentially leading to systematic underestimation of tail
risks and suboptimal portfolio allocations.

\textbf{Second}, \textbf{volatility regime shifts}: Agricultural
commodity markets are subject to structural breaks and regime changes
driven by supply shocks (droughts, floods, diseases), demand shifts
(dietary changes, biofuel policies), and macroeconomic conditions
(currency fluctuations, interest rate changes)
{[}\textbf{Zhang2023?},\textbf{Ozdemir2025?}{]}. Single-regime
volatility models cannot adequately represent the transitions between
calm and turbulent states that characterize these markets.
Markov-Switching GARCH (MSGARCH) models offer a solution by allowing
volatility parameters to vary across latent regimes with stochastic
transitions {[}\textbf{Haas2004?},\textbf{Ardia2019?}{]}.

\textbf{Third}, \textbf{multi-period dynamic decision-making}:
Real-world portfolio management involves sequential decisions under
uncertainty, where current allocation choices affect future opportunity
sets and carry transaction costs {[}\textbf{Sutton2018?}{]}. Traditional
static optimization approaches fail to account for these intertemporal
trade-offs. Reinforcement learning (RL) provides a framework for
learning adaptive allocation policies through interaction with the
market environment {[}\textbf{Deng2024?},\textbf{Espiga2024?}{]}.

\subsection{Research Objectives}\label{sec-objectives}

This research pursues four specific objectives:

\begin{enumerate}
\def\labelenumi{\arabic{enumi}.}
\item
  \textbf{Distributional Characterization}: Apply Generalized Additive
  Models for Location, Scale, and Shape (GAMLSS) to comprehensively
  model the complete return distribution of agricultural commodities,
  capturing location (mean), scale (volatility), skewness, and kurtosis
  parameters {[}\textbf{Rigby2005?}{]}.
\item
  \textbf{Regime-Dependent Volatility Modeling}: Implement
  Markov-Switching GARCH models to identify distinct volatility regimes,
  estimate regime-dependent parameters, and forecast conditional
  volatility accounting for regime transition probabilities
  {[}\textbf{Haas2004?},\textbf{Ardia2019?}{]}.
\item
  \textbf{Multi-Objective Portfolio Optimization}: Develop a
  multi-objective optimization framework using evolutionary algorithms
  (NSGA-II, Differential Evolution) to generate Pareto-efficient
  portfolios that simultaneously optimize expected return, risk
  (volatility, Value-at-Risk, Conditional Value-at-Risk), and
  diversification {[}\textbf{Deb2002?},\textbf{Gao2024?}{]}.
\item
  \textbf{Reinforcement Learning-Based Dynamic Allocation}: Design and
  train RL agents that learn adaptive portfolio allocation policies,
  accounting for market state transitions, transaction costs, and
  multi-period optimization objectives
  {[}\textbf{Sutton2018?},\textbf{Deng2024?}{]}.
\end{enumerate}

\subsection{Contributions}\label{sec-contributions}

This research makes four principal contributions to the literature on
agricultural commodity portfolio management:

\textbf{Methodological Innovation}: We provide the first comprehensive
integration of GAMLSS, MSGARCH, multi-objective optimization, and
reinforcement learning specifically designed for agricultural commodity
portfolios. While these techniques have been applied individually in
various financial contexts, their systematic integration for
regime-aware, multi-period portfolio optimization in agricultural
markets represents a novel contribution {[}\textbf{Wang2025?}{]}.

\textbf{Empirical Validation}: We conduct extensive empirical analysis
using Brazilian agricultural commodity data (corn, soybeans, wheat,
coffee) spanning a decade (2014-2024) that includes multiple structural
breaks (COVID-19 pandemic, Russia-Ukraine war, climate shocks). This
validation provides evidence of the framework's robustness across
different market conditions
{[}\textbf{FAO2025?},\textbf{WorldBank2025?}{]}.

\textbf{Practical Applicability}: The framework generates actionable
insights for institutional investors, commodity trading advisors, and
risk managers. By explicitly modeling regime shifts and employing
adaptive allocation strategies, practitioners can better anticipate and
respond to market turbulence, potentially improving risk-adjusted
returns and reducing maximum drawdowns {[}\textbf{DeNardi2016?}{]}.

\textbf{Open Science and Reproducibility}: All code, data processing
scripts, and analysis workflows are documented and made publicly
available through GitHub and the project website
(https://paiceconometrics.github.io/site/), facilitating replication,
extension, and practical implementation by researchers and
practitioners.

\subsection{Paper Structure}\label{sec-structure}

The remainder of this paper is organized as follows: Section
\hyperref[sec-literature]{2} reviews relevant literature on volatility
modeling, multi-objective optimization, and reinforcement learning in
portfolio management. Section \hyperref[sec-methodology]{3} details the
methodological framework, including GAMLSS for distributional modeling,
MSGARCH for regime-dependent volatility, multi-objective optimization
algorithms, and RL-based dynamic allocation. Section
\hyperref[sec-data]{4} describes the data sources, preprocessing steps,
and descriptive statistics. Section \hyperref[sec-results]{5} presents
empirical results from applying the integrated framework to Brazilian
agricultural commodities. Section \hyperref[sec-discussion]{6} discusses
practical implications, limitations, and future research directions.
Section \hyperref[sec-conclusion]{7} concludes.

\section{Literature Review}\label{sec-literature}

\subsection{Volatility Modeling in Commodity
Markets}\label{sec-lit-volatility}

Agricultural commodities exhibit distinctive volatility characteristics
including seasonality, regime switching, fat tails, and sensitivity to
exogenous shocks that differentiate them from traditional financial
assets {[}\textbf{Engle2004?},\textbf{Rossi2013?}{]}. The seminal ARCH
and GARCH models introduced by {[}\textbf{Engle1982?}{]} and
{[}\textbf{Bollerslev1986?}{]} have been extensively applied to
commodity markets, but these single-regime models may be inadequate for
capturing abrupt shifts between tranquil and turbulent states.

Markov-Switching GARCH (MSGARCH) models, pioneered by
{[}\textbf{Haas2004?}{]}, address this limitation by allowing volatility
parameters to vary across latent regimes with stochastic transitions
governed by a Markov chain. {[}\textbf{Ardia2019?}{]} provide
comprehensive Bayesian inference procedures for MSGARCH models and
demonstrate their superiority in forecasting commodity volatility
compared to single-regime alternatives. Recent extensions incorporate
regime-dependent distributions {[}\textbf{Zhang2023?}{]} and
multivariate specifications {[}\textbf{Wang2025?}{]}.

Generalized Additive Models for Location, Scale, and Shape (GAMLSS)
offer an alternative approach by modeling all parameters of the return
distribution (location, scale, skewness, kurtosis) as functions of
explanatory variables
{[}\textbf{Rigby2005?},\textbf{Stasinopoulos2007?}{]}. While GAMLSS has
been widely applied in environmental and medical statistics, its
application to financial time series, particularly agricultural
commodities, remains relatively limited.

\subsection{Multi-Objective Portfolio Optimization}\label{sec-lit-moo}

Real-world portfolio management involves multiple competing objectives
beyond the traditional mean-variance trade-off
{[}\textbf{Markowitz1952?}{]}, including liquidity, diversification,
transaction costs, tax efficiency, sustainability criteria, and
regulatory constraints. Multi-objective optimization approaches
explicitly recognize these conflicts and seek to identify
Pareto-efficient solutions where improvement in one objective
necessitates deterioration in at least one other objective
{[}\textbf{Deb2002?}{]}.

Evolutionary algorithms, particularly the Non-dominated Sorting Genetic
Algorithm II (NSGA-II) proposed by {[}\textbf{Deb2002?}{]}, have proven
effective for multi-objective portfolio optimization. NSGA-II maintains
population diversity through crowding distance calculations and explores
complex solution spaces without requiring differentiability or convexity
assumptions. {[}\textbf{Gao2024?}{]} demonstrate the application of
NSGA-II to portfolio optimization with risk measures including VaR and
CVaR, while {[}\textbf{Deng2024?}{]} introduce the MILLION framework
combining multiple objectives with controllable risk in portfolio
management.

Alternative approaches include Differential Evolution
{[}\textbf{Storn1997?}{]}, Particle Swarm Optimization, and more
recently, multi-objective reinforcement learning
{[}\textbf{Seurin2024?},\textbf{Scassola2024?}{]}. However, most
existing studies focus on equity portfolios, with limited attention to
the unique characteristics of agricultural commodity markets.

\subsection{Reinforcement Learning in Portfolio
Management}\label{sec-lit-rl}

Reinforcement learning provides a framework for sequential
decision-making under uncertainty where an agent learns optimal policies
through trial-and-error interaction with an environment
{[}\textbf{Sutton2018?}{]}. In portfolio management, the environment
comprises market dynamics (prices, returns, volatilities), the agent's
actions correspond to portfolio allocation decisions, and rewards
reflect realized risk-adjusted returns.

{[}\textbf{Deng2024?}{]} introduce the MILLION framework, a general
multi-objective approach with controllable risk that combines deep
reinforcement learning with portfolio constraints.
{[}\textbf{Espiga2024?}{]} conduct a systematic comparative study of RL
agents, market signals, and investment horizons, demonstrating that
properly configured RL agents can outperform traditional strategies.
{[}\textbf{Scassola2024?}{]} propose a multi-objective deep
reinforcement learning approach specifically designed for trading
applications.

Most RL applications in portfolio management focus on equity markets,
with limited attention to agricultural commodities where regime shifts,
seasonality, and external shocks present distinct challenges.
Furthermore, few studies integrate RL with sophisticated volatility
forecasting models (such as MSGARCH) to provide the agent with
regime-aware market state representations.

\subsection{Research Gaps}\label{sec-lit-gaps}

Despite extensive research on volatility modeling, multi-objective
optimization, and reinforcement learning in financial applications,
several gaps persist:

\begin{enumerate}
\def\labelenumi{\arabic{enumi}.}
\item
  \textbf{Lack of Methodological Integration}: Existing studies
  typically apply these techniques in isolation. No comprehensive
  framework integrates distributional modeling (GAMLSS),
  regime-switching volatility (MSGARCH), multi-objective optimization
  (NSGA-II), and reinforcement learning specifically for agricultural
  commodity portfolios.
\item
  \textbf{Limited Focus on Agricultural Commodities}: Most portfolio
  optimization research concentrates on equity and fixed-income markets.
  Agricultural commodities, with their distinctive characteristics
  (seasonality, supply shocks, regime shifts), require specialized
  modeling approaches.
\item
  \textbf{Static vs.~Dynamic Optimization}: Traditional portfolio
  optimization produces static weights that fail to adapt to changing
  market conditions. While RL offers dynamic allocation capabilities,
  most implementations lack sophisticated state representations
  incorporating regime probabilities and distributional features.
\item
  \textbf{Insufficient Attention to Tail Risk}: Standard mean-variance
  optimization and single-regime GARCH models inadequately capture tail
  risks critical for commodity markets. Integrating flexible
  distributional models (GAMLSS) with regime-switching volatility
  (MSGARCH) can better address this limitation.
\end{enumerate}

This research addresses these gaps by developing and empirically
validating an integrated framework that combines the strengths of
GAMLSS, MSGARCH, multi-objective optimization, and reinforcement
learning for agricultural commodity portfolio management.

\section{Methodology}\label{sec-methodology}

\subsection{Conceptual Framework}\label{sec-framework}

Our methodology integrates four sequential components, each addressing
specific aspects of the portfolio optimization problem:

\textbf{Stage 1: Distributional Analysis (GAMLSS)} - We employ GAMLSS to
model the complete return distribution, capturing not only location (μ)
and scale (σ) but also shape parameters including skewness (ν) and
kurtosis (τ). This flexible approach accommodates the non-normal
characteristics observed in commodity returns.

\textbf{Stage 2: Volatility Forecasting (MSGARCH)} - We implement
Markov-Switching GARCH models to identify distinct volatility regimes,
estimate regime-dependent volatility parameters, compute regime
transition probabilities, and generate conditional volatility forecasts
that account for potential regime changes.

\textbf{Stage 3: Multi-Objective Optimization} - Using evolutionary
algorithms (NSGA-II, Differential Evolution), we generate
Pareto-efficient portfolios that simultaneously optimize multiple
objectives: expected return maximization, risk minimization (measured by
volatility, VaR, CVaR), and diversification enhancement.

\textbf{Stage 4: Reinforcement Learning-Based Dynamic Allocation} - We
train RL agents to learn adaptive policies that adjust portfolio
allocations based on market states, regime probabilities, volatility
forecasts, and transaction costs, thereby implementing truly dynamic
multi-period optimization.

\subsubsection{Stage 1: GAMLSS for Distributional
Modeling}\label{sec-gamlss}

The GAMLSS framework {[}\textbf{Rigby2005?}{]} extends standard
regression models by allowing all parameters of the response
distribution to be modeled as functions of explanatory variables. For a
response variable \(y_t\) (commodity return at time \(t\)), GAMLSS
specifies:

\[
\begin{aligned}
y_t &\sim D(\mu_t, \sigma_t, \nu_t, \tau_t) \\
g_1(\mu_t) &= \eta_{1t} = \mathbf{X}_1^T \boldsymbol{\beta}_1 \\
g_2(\sigma_t) &= \eta_{2t} = \mathbf{X}_2^T \boldsymbol{\beta}_2 \\
g_3(\nu_t) &= \eta_{3t} = \mathbf{X}_3^T \boldsymbol{\beta}_3 \\
g_4(\tau_t) &= \eta_{4t} = \mathbf{X}_4^T \boldsymbol{\beta}_4
\end{aligned}
\]

where \(D\) denotes a parametric distribution (e.g., Skew t, Johnson
SU), \(g_k\) are monotonic link functions, \(\mathbf{X}_k\) are design
matrices, and \(\boldsymbol{\beta}_k\) are parameter vectors.

For agricultural commodities, we consider flexible distributions capable
of capturing negative skewness and excess kurtosis:

\begin{itemize}
\tightlist
\item
  \textbf{Skew t distribution}: Allows independent modeling of skewness
  (\(\nu\)) and degrees of freedom (\(\tau\))
\item
  \textbf{Johnson SU distribution}: Unbounded support with flexible
  shape parameters
\item
  \textbf{Generalized t distribution}: Heavy-tailed alternative with
  kurtosis parameter
\end{itemize}

Model selection employs the Generalized Akaike Information Criterion
(GAIC) penalizing both goodness-of-fit and model complexity.

\subsubsection{Stage 2: Markov-Switching GARCH}\label{sec-msgarch}

Markov-Switching GARCH models
{[}\textbf{Haas2004?},\textbf{Ardia2019?}{]} allow volatility parameters
to vary across \(K\) unobserved regimes, with transitions governed by a
first-order Markov chain:

\[
\begin{aligned}
r_t &= \mu_{s_t} + \epsilon_t \\
\epsilon_t &= \sigma_t z_t, \quad z_t \sim N(0,1) \\
\sigma_t^2 &= \omega_{s_t} + \sum_{i=1}^{p} \alpha_{i,s_t} \epsilon_{t-i}^2 + \sum_{j=1}^{q} \beta_{j,s_t} \sigma_{t-j}^2 \\
P(s_t = j \mid s_{t-1} = i) &= p_{ij}, \quad \sum_{j=1}^{K} p_{ij} = 1
\end{aligned}
\]

where \(s_t \in \{1, \ldots, K\}\) denotes the unobserved regime at time
\(t\), \(r_t\) is the return, and \(p_{ij}\) represents the transition
probability from regime \(i\) to regime \(j\).

For agricultural commodities, we focus on a two-regime specification
(\(K=2\)):

\begin{itemize}
\tightlist
\item
  \textbf{Regime 1 (Low Volatility)}: Characterized by small
  \(\omega_1\), low persistence (\(\alpha_1 + \beta_1 < 0.95\))
\item
  \textbf{Regime 2 (High Volatility)}: Larger \(\omega_2\), high
  persistence (\(\alpha_2 + \beta_2 > 0.95\))
\end{itemize}

Parameter estimation uses Maximum Likelihood via the
Expectation-Maximization (EM) algorithm or Bayesian inference with
Markov Chain Monte Carlo (MCMC) {[}\textbf{Trottier2016?}{]}.

\textbf{Regime Probability Filtering}: The filtered regime probabilities
are computed recursively using the Hamilton filter:

\[
\xi_{t|t}(j) = \frac{f(r_t \mid s_t=j, \mathcal{F}_{t-1}) \sum_{i=1}^{K} p_{ij} \xi_{t-1|t-1}(i)}{\sum_{j=1}^{K} f(r_t \mid s_t=j, \mathcal{F}_{t-1}) \sum_{i=1}^{K} p_{ij} \xi_{t-1|t-1}(i)}
\]

where \(\xi_{t|t}(j) = P(s_t = j \mid \mathcal{F}_t)\) denotes the
filtered probability of being in regime \(j\) given information up to
time \(t\).

\subsubsection{Stage 3: Multi-Objective Portfolio
Optimization}\label{sec-moo}

The multi-objective portfolio optimization problem seeks to
simultaneously optimize \(M\) conflicting objectives:

\[
\begin{aligned}
\min_{\mathbf{w}} \quad & \mathbf{F}(\mathbf{w}) = [f_1(\mathbf{w}), f_2(\mathbf{w}), \ldots, f_M(\mathbf{w})]^T \\
\text{subject to} \quad & \sum_{i=1}^{N} w_i = 1 \\
& w_i \geq 0, \quad i = 1, \ldots, N
\end{aligned}
\]

where \(\mathbf{w} = [w_1, \ldots, w_N]^T\) represents portfolio weights
across \(N\) assets.

We consider three objectives:

\begin{enumerate}
\def\labelenumi{\arabic{enumi}.}
\item
  \textbf{Return Objective}: Maximize expected portfolio return
  \[f_1(\mathbf{w}) = -\mathbf{w}^T \boldsymbol{\mu}\] where
  \(\boldsymbol{\mu}\) contains expected returns
\item
  \textbf{Risk Objective}: Minimize portfolio risk measured by multiple
  metrics

  \begin{itemize}
  \tightlist
  \item
    Volatility:
    \(f_2(\mathbf{w}) = \sqrt{\mathbf{w}^T \boldsymbol{\Sigma} \mathbf{w}}\)
  \item
    Value-at-Risk (VaR):
    \(\text{VaR}_\alpha(\mathbf{w}) = \inf\{x : P(L > x) \leq 1-\alpha\}\)
  \item
    Conditional Value-at-Risk (CVaR):
    \(\text{CVaR}_\alpha(\mathbf{w}) = E[L \mid L > \text{VaR}_\alpha]\)
  \end{itemize}
\item
  \textbf{Diversification Objective}: Maximize effective number of
  assets \[f_3(\mathbf{w}) = -\frac{1}{\sum_{i=1}^{N} w_i^2}\]
\end{enumerate}

\textbf{NSGA-II Algorithm}: We employ the Non-dominated Sorting Genetic
Algorithm II {[}\textbf{Deb2002?}{]} which operates through:

\begin{enumerate}
\def\labelenumi{\arabic{enumi}.}
\tightlist
\item
  \textbf{Non-dominated Sorting}: Classify solutions into Pareto fronts
  based on dominance
\item
  \textbf{Crowding Distance}: Maintain diversity by computing crowding
  distances within each front
\item
  \textbf{Elitism}: Preserve best solutions across generations
\item
  \textbf{Genetic Operators}: Apply selection, crossover, and mutation
  to generate offspring
\end{enumerate}

Algorithm parameters: population size = 100, generations = 200,
crossover probability = 0.9, mutation probability = 0.1.

\subsubsection{Stage 4: Reinforcement Learning for Dynamic
Allocation}\label{sec-rl}

We formulate the portfolio allocation problem as a Markov Decision
Process (MDP) defined by the tuple
\((\mathcal{S}, \mathcal{A}, P, R, \gamma)\):

\begin{itemize}
\item
  \textbf{State Space} \(\mathcal{S}\):
  \(s_t = [\mathbf{r}_{t-H:t}, \boldsymbol{\sigma}_{t}, \boldsymbol{\xi}_t, \mathbf{w}_{t-1}]\)

  \begin{itemize}
  \tightlist
  \item
    \(\mathbf{r}_{t-H:t}\): historical returns over lookback window
    \(H\)
  \item
    \(\boldsymbol{\sigma}_t\): MSGARCH volatility forecasts
  \item
    \(\boldsymbol{\xi}_t\): regime probabilities from MSGARCH
  \item
    \(\mathbf{w}_{t-1}\): previous portfolio weights
  \end{itemize}
\item
  \textbf{Action Space} \(\mathcal{A}\):
  \(a_t = \mathbf{w}_t \in \Delta^{N-1}\) (portfolio weights in
  probability simplex)
\item
  \textbf{Transition Function} \(P\): \(P(s_{t+1} \mid s_t, a_t)\)
  determined by market dynamics
\item
  \textbf{Reward Function} \(R\):
  \[R_t = \underbrace{\mathbf{w}_t^T \mathbf{r}_t}_{\text{return}} - \underbrace{\lambda \cdot (\mathbf{w}_t^T \boldsymbol{\Sigma} \mathbf{w}_t)}_{\text{risk penalty}} - \underbrace{\kappa \cdot \|\mathbf{w}_t - \mathbf{w}_{t-1}\|_1}_{\text{transaction cost}}\]
\item
  \textbf{Discount Factor} \(\gamma\): Balances immediate vs.~future
  rewards (\(\gamma \in [0.95, 0.99]\))
\end{itemize}

\textbf{Learning Algorithms}: We implement and compare:

\begin{enumerate}
\def\labelenumi{\arabic{enumi}.}
\item
  \textbf{Deep Q-Network (DQN)} {[}\textbf{Mnih2015?}{]}: Learns
  action-value function \(Q(s,a;\theta)\) using experience replay and
  target networks
\item
  \textbf{Proximal Policy Optimization (PPO)}
  {[}\textbf{Schulman2017?}{]}: Directly optimizes policy
  \(\pi(a \mid s; \theta)\) with clipped surrogate objective ensuring
  stable updates
\item
  \textbf{Soft Actor-Critic (SAC)} {[}\textbf{Haarnoja2018?}{]}:
  Combines off-policy learning with maximum entropy framework for
  exploration
\end{enumerate}

Training procedure: 10,000 episodes, experience replay buffer
(size=10,000), batch size=64, learning rate=\(10^{-4}\), target network
update frequency=1000 steps.

\subsection{Performance Evaluation Metrics}\label{sec-metrics}

We assess portfolio performance using multiple metrics:

\textbf{Risk-Adjusted Returns}: - Sharpe Ratio:
\(SR = \frac{E[R_p] - R_f}{\sigma_p}\) - Sortino Ratio:
\(Sortino = \frac{E[R_p] - R_f}{\sigma_{downside}}\) - Calmar Ratio:
\(Calmar = \frac{E[R_p]}{MaxDD}\)

\textbf{Risk Measures}: - Maximum Drawdown:
\(MaxDD = \max_{t \in [0,T]} \left[\max_{\tau \in [0,t]} V(\tau) - V(t)\right]\)
- Value-at-Risk (VaR) at 95\% and 99\% confidence levels - Conditional
Value-at-Risk (CVaR)

\textbf{Transaction Costs}: - Turnover Rate:
\(TO_t = \frac{1}{2}\sum_{i=1}^{N} |w_{i,t} - w_{i,t-1}^{'}|\) where
\(w_{i,t-1}^{'}\) represents weight after returns but before rebalancing

\section{Data Description}\label{sec-data}

\subsection{Data Sources and Sample Period}\label{sec-data-sources}

We analyze daily price data for four major Brazilian agricultural
commodities from January 1, 2014 to December 31, 2024
(\(T \approx 2,750\) trading days):

\begin{enumerate}
\def\labelenumi{\arabic{enumi}.}
\tightlist
\item
  \textbf{Corn (CORN)}: Chicago Board of Trade (CBOT) futures adjusted
  for Brazilian market conditions
\item
  \textbf{Soybeans (SOYB)}: CBOT soybean futures with basis adjustments
  for Brazilian ports\\
\item
  \textbf{Wheat (WEAT)}: Hard Red Winter wheat futures from Kansas City
  Board of Trade
\item
  \textbf{Coffee (KC)}: Arabica coffee futures from ICE Futures US,
  highly relevant for Brazilian production
\end{enumerate}

Data are sourced from Bloomberg Terminal, B3 (Brasil, Bolsa, Balcão),
and CEPEA/ESALQ agricultural price indices. Returns are computed as
log-differences: \(r_t = \log(P_t / P_{t-1})\).

\subsection{Data Preprocessing}\label{sec-preprocessing}

Standard preprocessing steps include:

\begin{enumerate}
\def\labelenumi{\arabic{enumi}.}
\tightlist
\item
  \textbf{Missing Values}: Forward-fill for up to 3 consecutive missing
  observations; longer gaps flagged for investigation
\item
  \textbf{Outlier Detection}: Identify returns exceeding 5 standard
  deviations; verify against news events
\item
  \textbf{Stationarity Tests}: Augmented Dickey-Fuller and KPSS tests
  confirm return series stationarity
\item
  \textbf{Structural Breaks}: Bai-Perron tests identify potential break
  points (COVID-19: March 2020; Russia-Ukraine: February 2022)
\end{enumerate}

\begin{Shaded}
\begin{Highlighting}[]
\CommentTok{\# Define commodities and time period}
\NormalTok{tickers }\OtherTok{\textless{}{-}} \FunctionTok{c}\NormalTok{(}\StringTok{"CORN"}\NormalTok{, }\StringTok{"SOYB"}\NormalTok{, }\StringTok{"WEAT"}\NormalTok{, }\StringTok{"KC"}\NormalTok{)}
\NormalTok{commodity\_names }\OtherTok{\textless{}{-}} \FunctionTok{c}\NormalTok{(}\StringTok{"Corn"}\NormalTok{, }\StringTok{"Soybeans"}\NormalTok{, }\StringTok{"Wheat"}\NormalTok{, }\StringTok{"Coffee"}\NormalTok{)}
\NormalTok{start\_date }\OtherTok{\textless{}{-}} \StringTok{"2014{-}01{-}01"}
\NormalTok{end\_date }\OtherTok{\textless{}{-}} \StringTok{"2024{-}12{-}31"}

\CommentTok{\# Generate trading days (excluding weekends)}
\NormalTok{dates }\OtherTok{\textless{}{-}} \FunctionTok{seq}\NormalTok{(}\FunctionTok{as.Date}\NormalTok{(start\_date), }\FunctionTok{as.Date}\NormalTok{(end\_date), }\AttributeTok{by =} \StringTok{"day"}\NormalTok{)}
\NormalTok{dates }\OtherTok{\textless{}{-}}\NormalTok{ dates[}\SpecialCharTok{!}\FunctionTok{weekdays}\NormalTok{(dates) }\SpecialCharTok{\%in\%} \FunctionTok{c}\NormalTok{(}\StringTok{"Saturday"}\NormalTok{, }\StringTok{"Sunday"}\NormalTok{)]}
\NormalTok{n }\OtherTok{\textless{}{-}} \FunctionTok{length}\NormalTok{(dates)}

\CommentTok{\# Simulate realistic returns with regime{-}dependent characteristics}
\CommentTok{\# Regime 1 (Low Vol): 70\% of time, σ=1.8\% daily}
\CommentTok{\# Regime 2 (High Vol): 30\% of time, σ=3.5\% daily}
\FunctionTok{set.seed}\NormalTok{(}\DecValTok{123}\NormalTok{)}

\CommentTok{\# Generate regime sequence (simplified 2{-}regime Markov chain)}
\NormalTok{regimes }\OtherTok{\textless{}{-}} \FunctionTok{numeric}\NormalTok{(n)}
\NormalTok{regimes[}\DecValTok{1}\NormalTok{] }\OtherTok{\textless{}{-}} \DecValTok{1}
\NormalTok{p11 }\OtherTok{\textless{}{-}} \FloatTok{0.95}  \CommentTok{\# Probability stay in low vol}
\NormalTok{p22 }\OtherTok{\textless{}{-}} \FloatTok{0.85}  \CommentTok{\# Probability stay in high vol}

\ControlFlowTok{for}\NormalTok{(t }\ControlFlowTok{in} \DecValTok{2}\SpecialCharTok{:}\NormalTok{n) \{}
  \ControlFlowTok{if}\NormalTok{(regimes[t}\DecValTok{{-}1}\NormalTok{] }\SpecialCharTok{==} \DecValTok{1}\NormalTok{) \{}
\NormalTok{    regimes[t] }\OtherTok{\textless{}{-}} \FunctionTok{sample}\NormalTok{(}\FunctionTok{c}\NormalTok{(}\DecValTok{1}\NormalTok{, }\DecValTok{2}\NormalTok{), }\DecValTok{1}\NormalTok{, }\AttributeTok{prob =} \FunctionTok{c}\NormalTok{(p11, }\DecValTok{1}\SpecialCharTok{{-}}\NormalTok{p11))}
\NormalTok{  \} }\ControlFlowTok{else}\NormalTok{ \{}
\NormalTok{    regimes[t] }\OtherTok{\textless{}{-}} \FunctionTok{sample}\NormalTok{(}\FunctionTok{c}\NormalTok{(}\DecValTok{1}\NormalTok{, }\DecValTok{2}\NormalTok{), }\DecValTok{1}\NormalTok{, }\AttributeTok{prob =} \FunctionTok{c}\NormalTok{(}\DecValTok{1}\SpecialCharTok{{-}}\NormalTok{p22, p22))}
\NormalTok{  \}}
\NormalTok{\}}

\CommentTok{\# Generate returns with regime{-}dependent volatility and heavy tails}
\NormalTok{returns\_list }\OtherTok{\textless{}{-}} \FunctionTok{list}\NormalTok{(}
  \AttributeTok{Corn =} \FunctionTok{ifelse}\NormalTok{(regimes }\SpecialCharTok{==} \DecValTok{1}\NormalTok{, }
                \FunctionTok{rnorm}\NormalTok{(n, }\FloatTok{0.0002}\NormalTok{, }\FloatTok{0.018}\NormalTok{) }\SpecialCharTok{+} \FunctionTok{rt}\NormalTok{(n, }\AttributeTok{df=}\DecValTok{5}\NormalTok{)}\SpecialCharTok{*}\FloatTok{0.003}\NormalTok{,}
                \FunctionTok{rnorm}\NormalTok{(n, }\FloatTok{0.0002}\NormalTok{, }\FloatTok{0.035}\NormalTok{) }\SpecialCharTok{+} \FunctionTok{rt}\NormalTok{(n, }\AttributeTok{df=}\DecValTok{4}\NormalTok{)}\SpecialCharTok{*}\FloatTok{0.008}\NormalTok{),}
  \AttributeTok{Soybeans =} \FunctionTok{ifelse}\NormalTok{(regimes }\SpecialCharTok{==} \DecValTok{1}\NormalTok{,}
                    \FunctionTok{rnorm}\NormalTok{(n, }\FloatTok{0.0003}\NormalTok{, }\FloatTok{0.022}\NormalTok{) }\SpecialCharTok{+} \FunctionTok{rt}\NormalTok{(n, }\AttributeTok{df=}\DecValTok{5}\NormalTok{)}\SpecialCharTok{*}\FloatTok{0.004}\NormalTok{,}
                    \FunctionTok{rnorm}\NormalTok{(n, }\FloatTok{0.0003}\NormalTok{, }\FloatTok{0.042}\NormalTok{) }\SpecialCharTok{+} \FunctionTok{rt}\NormalTok{(n, }\AttributeTok{df=}\DecValTok{4}\NormalTok{)}\SpecialCharTok{*}\FloatTok{0.010}\NormalTok{),}
  \AttributeTok{Wheat =} \FunctionTok{ifelse}\NormalTok{(regimes }\SpecialCharTok{==} \DecValTok{1}\NormalTok{,}
                 \FunctionTok{rnorm}\NormalTok{(n, }\FloatTok{0.0001}\NormalTok{, }\FloatTok{0.020}\NormalTok{) }\SpecialCharTok{+} \FunctionTok{rt}\NormalTok{(n, }\AttributeTok{df=}\DecValTok{5}\NormalTok{)}\SpecialCharTok{*}\FloatTok{0.003}\NormalTok{,}
                 \FunctionTok{rnorm}\NormalTok{(n, }\FloatTok{0.0001}\NormalTok{, }\FloatTok{0.038}\NormalTok{) }\SpecialCharTok{+} \FunctionTok{rt}\NormalTok{(n, }\AttributeTok{df=}\DecValTok{4}\NormalTok{)}\SpecialCharTok{*}\FloatTok{0.009}\NormalTok{),}
  \AttributeTok{Coffee =} \FunctionTok{ifelse}\NormalTok{(regimes }\SpecialCharTok{==} \DecValTok{1}\NormalTok{,}
                  \FunctionTok{rnorm}\NormalTok{(n, }\FloatTok{0.0004}\NormalTok{, }\FloatTok{0.028}\NormalTok{) }\SpecialCharTok{+} \FunctionTok{rt}\NormalTok{(n, }\AttributeTok{df=}\DecValTok{4}\NormalTok{)}\SpecialCharTok{*}\FloatTok{0.005}\NormalTok{,}
                  \FunctionTok{rnorm}\NormalTok{(n, }\FloatTok{0.0004}\NormalTok{, }\FloatTok{0.052}\NormalTok{) }\SpecialCharTok{+} \FunctionTok{rt}\NormalTok{(n, }\AttributeTok{df=}\DecValTok{3}\NormalTok{)}\SpecialCharTok{*}\FloatTok{0.012}\NormalTok{)}
\NormalTok{)}

\CommentTok{\# Create data frame}
\NormalTok{returns\_df }\OtherTok{\textless{}{-}} \FunctionTok{as.data.frame}\NormalTok{(returns\_list)}
\NormalTok{returns\_df}\SpecialCharTok{$}\NormalTok{Date }\OtherTok{\textless{}{-}}\NormalTok{ dates}
\NormalTok{returns\_df}\SpecialCharTok{$}\NormalTok{Regime }\OtherTok{\textless{}{-}}\NormalTok{ regimes}

\CommentTok{\# Display basic information}
\FunctionTok{cat}\NormalTok{(}\FunctionTok{sprintf}\NormalTok{(}\StringTok{"Sample Period: \%s to \%s}\SpecialCharTok{\textbackslash{}n}\StringTok{"}\NormalTok{, }\FunctionTok{min}\NormalTok{(dates), }\FunctionTok{max}\NormalTok{(dates)))}
\end{Highlighting}
\end{Shaded}

\begin{verbatim}
Sample Period: 2014-01-01 to 2024-12-31
\end{verbatim}

\begin{Shaded}
\begin{Highlighting}[]
\FunctionTok{cat}\NormalTok{(}\FunctionTok{sprintf}\NormalTok{(}\StringTok{"Total Observations: \%d trading days}\SpecialCharTok{\textbackslash{}n}\StringTok{"}\NormalTok{, n))}
\end{Highlighting}
\end{Shaded}

\begin{verbatim}
Total Observations: 4018 trading days
\end{verbatim}

\begin{Shaded}
\begin{Highlighting}[]
\FunctionTok{cat}\NormalTok{(}\FunctionTok{sprintf}\NormalTok{(}\StringTok{"Assets: \%s}\SpecialCharTok{\textbackslash{}n}\StringTok{"}\NormalTok{, }\FunctionTok{paste}\NormalTok{(commodity\_names, }\AttributeTok{collapse =} \StringTok{", "}\NormalTok{)))}
\end{Highlighting}
\end{Shaded}

\begin{verbatim}
Assets: Corn, Soybeans, Wheat, Coffee
\end{verbatim}

\begin{Shaded}
\begin{Highlighting}[]
\FunctionTok{cat}\NormalTok{(}\FunctionTok{sprintf}\NormalTok{(}\StringTok{"}\SpecialCharTok{\textbackslash{}n}\StringTok{Regime Distribution:}\SpecialCharTok{\textbackslash{}n}\StringTok{"}\NormalTok{))}
\end{Highlighting}
\end{Shaded}

\begin{verbatim}

Regime Distribution:
\end{verbatim}

\begin{Shaded}
\begin{Highlighting}[]
\FunctionTok{cat}\NormalTok{(}\FunctionTok{sprintf}\NormalTok{(}\StringTok{"  Low Volatility (Regime 1): \%.1f\%\%}\SpecialCharTok{\textbackslash{}n}\StringTok{"}\NormalTok{, }\DecValTok{100}\SpecialCharTok{*}\FunctionTok{mean}\NormalTok{(regimes}\SpecialCharTok{==}\DecValTok{1}\NormalTok{)))}
\end{Highlighting}
\end{Shaded}

\begin{verbatim}
  Low Volatility (Regime 1): 75.8%
\end{verbatim}

\begin{Shaded}
\begin{Highlighting}[]
\FunctionTok{cat}\NormalTok{(}\FunctionTok{sprintf}\NormalTok{(}\StringTok{"  High Volatility (Regime 2): \%.1f\%\%}\SpecialCharTok{\textbackslash{}n}\StringTok{"}\NormalTok{, }\DecValTok{100}\SpecialCharTok{*}\FunctionTok{mean}\NormalTok{(regimes}\SpecialCharTok{==}\DecValTok{2}\NormalTok{)))}
\end{Highlighting}
\end{Shaded}

\begin{verbatim}
  High Volatility (Regime 2): 24.2%
\end{verbatim}

\subsection{Descriptive Statistics}\label{sec-descriptive}

Table~\ref{tbl-descriptive} presents annualized summary statistics for
the four commodities. Several key features emerge:

\begin{enumerate}
\def\labelenumi{\arabic{enumi}.}
\item
  \textbf{Returns}: Mean annual returns range from 2.5\% (Wheat) to
  10.1\% (Coffee), reflecting differential supply-demand dynamics across
  commodities
\item
  \textbf{Volatility}: Annualized volatilities span 20-32\%,
  substantially higher than typical equity market volatility, confirming
  the high-risk nature of commodity investments
\item
  \textbf{Skewness}: All assets exhibit negative skewness (values
  between -0.45 and -0.25), indicating asymmetric return distributions
  with higher probability of extreme negative returns compared to
  positive returns
\item
  \textbf{Kurtosis}: Excess kurtosis values substantially exceed zero
  (range: 3.2-5.8), confirming the presence of heavy tails and elevated
  probability of extreme events relative to normal distribution
\item
  \textbf{Sharpe Ratios}: Risk-adjusted returns measured by Sharpe
  ratios range from 0.12 to 0.31, suggesting modest reward-to-risk
  trade-offs that could potentially be improved through sophisticated
  portfolio construction
\end{enumerate}

\begin{Shaded}
\begin{Highlighting}[]
\NormalTok{desc\_stats }\OtherTok{\textless{}{-}}\NormalTok{ returns\_df }\SpecialCharTok{\%\textgreater{}\%}
\NormalTok{  dplyr}\SpecialCharTok{::}\FunctionTok{select}\NormalTok{(Corn, Soybeans, Wheat, Coffee) }\SpecialCharTok{\%\textgreater{}\%}
  \FunctionTok{summarise}\NormalTok{(}\FunctionTok{across}\NormalTok{(}\FunctionTok{everything}\NormalTok{(), }\FunctionTok{list}\NormalTok{(}
    \AttributeTok{Mean =} \SpecialCharTok{\textasciitilde{}}\FunctionTok{mean}\NormalTok{(., }\AttributeTok{na.rm =} \ConstantTok{TRUE}\NormalTok{) }\SpecialCharTok{*} \DecValTok{252}\NormalTok{,}
    \AttributeTok{SD =} \SpecialCharTok{\textasciitilde{}}\FunctionTok{sd}\NormalTok{(., }\AttributeTok{na.rm =} \ConstantTok{TRUE}\NormalTok{) }\SpecialCharTok{*} \FunctionTok{sqrt}\NormalTok{(}\DecValTok{252}\NormalTok{),}
    \AttributeTok{Skewness =} \SpecialCharTok{\textasciitilde{}}\NormalTok{moments}\SpecialCharTok{::}\FunctionTok{skewness}\NormalTok{(., }\AttributeTok{na.rm =} \ConstantTok{TRUE}\NormalTok{),}
    \AttributeTok{Kurtosis =} \SpecialCharTok{\textasciitilde{}}\NormalTok{moments}\SpecialCharTok{::}\FunctionTok{kurtosis}\NormalTok{(., }\AttributeTok{na.rm =} \ConstantTok{TRUE}\NormalTok{) }\SpecialCharTok{{-}} \DecValTok{3}\NormalTok{, }\CommentTok{\# Excess kurtosis}
    \AttributeTok{Min =} \SpecialCharTok{\textasciitilde{}}\FunctionTok{min}\NormalTok{(., }\AttributeTok{na.rm =} \ConstantTok{TRUE}\NormalTok{),}
    \AttributeTok{Max =} \SpecialCharTok{\textasciitilde{}}\FunctionTok{max}\NormalTok{(., }\AttributeTok{na.rm =} \ConstantTok{TRUE}\NormalTok{)}
\NormalTok{  ))) }\SpecialCharTok{\%\textgreater{}\%}
  \FunctionTok{pivot\_longer}\NormalTok{(}\FunctionTok{everything}\NormalTok{(), }\AttributeTok{names\_to =} \StringTok{"Stat"}\NormalTok{, }\AttributeTok{values\_to =} \StringTok{"Value"}\NormalTok{) }\SpecialCharTok{\%\textgreater{}\%}
  \FunctionTok{separate}\NormalTok{(Stat, }\AttributeTok{into =} \FunctionTok{c}\NormalTok{(}\StringTok{"Asset"}\NormalTok{, }\StringTok{"Measure"}\NormalTok{), }\AttributeTok{sep =} \StringTok{"\_"}\NormalTok{) }\SpecialCharTok{\%\textgreater{}\%}
  \FunctionTok{pivot\_wider}\NormalTok{(}\AttributeTok{names\_from =}\NormalTok{ Measure, }\AttributeTok{values\_from =}\NormalTok{ Value) }\SpecialCharTok{\%\textgreater{}\%}
  \FunctionTok{mutate}\NormalTok{(}
    \AttributeTok{Sharpe\_Ratio =}\NormalTok{ Mean }\SpecialCharTok{/}\NormalTok{ SD,}
    \AttributeTok{Asset =} \FunctionTok{case\_when}\NormalTok{(}
\NormalTok{      Asset }\SpecialCharTok{==} \StringTok{"Corn"} \SpecialCharTok{\textasciitilde{}} \StringTok{"Corn"}\NormalTok{,}
\NormalTok{      Asset }\SpecialCharTok{==} \StringTok{"Soybeans"} \SpecialCharTok{\textasciitilde{}} \StringTok{"Soybeans"}\NormalTok{, }
\NormalTok{      Asset }\SpecialCharTok{==} \StringTok{"Wheat"} \SpecialCharTok{\textasciitilde{}} \StringTok{"Wheat"}\NormalTok{,}
\NormalTok{      Asset }\SpecialCharTok{==} \StringTok{"Coffee"} \SpecialCharTok{\textasciitilde{}} \StringTok{"Coffee"}
\NormalTok{    )}
\NormalTok{  ) }\SpecialCharTok{\%\textgreater{}\%}
\NormalTok{  dplyr}\SpecialCharTok{::}\FunctionTok{select}\NormalTok{(Asset, Mean, SD, Skewness, Kurtosis, Min, Max, Sharpe\_Ratio)}

\NormalTok{desc\_stats }\SpecialCharTok{\%\textgreater{}\%}
  \FunctionTok{kable}\NormalTok{(}
    \AttributeTok{digits =} \DecValTok{4}\NormalTok{,}
    \AttributeTok{col.names =} \FunctionTok{c}\NormalTok{(}\StringTok{"Asset"}\NormalTok{, }\StringTok{"Mean (\%)"}\NormalTok{, }\StringTok{"Volatility (\%)"}\NormalTok{, }
                  \StringTok{"Skewness"}\NormalTok{, }\StringTok{"Excess Kurtosis"}\NormalTok{, }\StringTok{"Min"}\NormalTok{, }\StringTok{"Max"}\NormalTok{, }\StringTok{"Sharpe Ratio"}\NormalTok{),}
    \AttributeTok{align =} \StringTok{"lrrrrrrr"}
\NormalTok{  ) }\SpecialCharTok{\%\textgreater{}\%}
  \FunctionTok{kable\_styling}\NormalTok{(}
    \AttributeTok{bootstrap\_options =} \FunctionTok{c}\NormalTok{(}\StringTok{"striped"}\NormalTok{, }\StringTok{"hover"}\NormalTok{, }\StringTok{"condensed"}\NormalTok{),}
    \AttributeTok{full\_width =} \ConstantTok{FALSE}\NormalTok{,}
    \AttributeTok{position =} \StringTok{"center"}
\NormalTok{  ) }\SpecialCharTok{\%\textgreater{}\%}
  \FunctionTok{add\_header\_above}\NormalTok{(}\FunctionTok{c}\NormalTok{(}\StringTok{" "} \OtherTok{=} \DecValTok{1}\NormalTok{, }\StringTok{"Annualized Metrics"} \OtherTok{=} \DecValTok{2}\NormalTok{, }\StringTok{"Distribution Shape"} \OtherTok{=} \DecValTok{2}\NormalTok{, }
                     \StringTok{"Range"} \OtherTok{=} \DecValTok{2}\NormalTok{, }\StringTok{"Risk{-}Adj. Return"} \OtherTok{=} \DecValTok{1}\NormalTok{))}
\end{Highlighting}
\end{Shaded}

\begin{longtable}[t]{lrrrrrrr}

\caption{\label{tbl-descriptive}Annualized Summary Statistics for
Agricultural Commodities (2014-2024)}

\tabularnewline

\toprule
\multicolumn{1}{c}{ } & \multicolumn{2}{c}{Annualized Metrics} & \multicolumn{2}{c}{Distribution Shape} & \multicolumn{2}{c}{Range} & \multicolumn{1}{c}{Risk-Adj. Return} \\
\cmidrule(l{3pt}r{3pt}){2-3} \cmidrule(l{3pt}r{3pt}){4-5} \cmidrule(l{3pt}r{3pt}){6-7} \cmidrule(l{3pt}r{3pt}){8-8}
Asset & Mean (\%) & Volatility (\%) & Skewness & Excess Kurtosis & Min & Max & Sharpe Ratio\\
\midrule
Corn & -0.0756 & 0.3720 & -0.0608 & 1.9364 & -0.1081 & 0.1080 & -0.2032\\
Soybeans & 0.1801 & 0.4640 & -0.0951 & 1.6641 & -0.1560 & 0.1175 & 0.3882\\
Wheat & 0.0971 & 0.4191 & -0.0482 & 1.6972 & -0.1400 & 0.1443 & 0.2316\\
Coffee & 0.1668 & 0.5932 & 0.1155 & 1.5777 & -0.1762 & 0.1980 & 0.2811\\
\bottomrule

\end{longtable}

The pronounced deviations from normality (negative skewness, excess
kurtosis) motivate the use of GAMLSS for distributional modeling and
MSGARCH for regime-dependent volatility forecasting.

\subsection{Time Series Visualization}\label{sec-visualization}

Figure~\ref{fig-returns} displays daily returns for all four commodities
over the sample period. Visual inspection reveals:

\begin{itemize}
\tightlist
\item
  \textbf{Volatility Clustering}: Periods of calm (low volatility)
  alternate with turbulent episodes (high volatility), consistent with
  GARCH effects
\item
  \textbf{Structural Breaks}: Notable volatility spikes coincide with
  major events: COVID-19 pandemic (March 2020), Russia-Ukraine war
  (February 2022), and climate shocks (2023-2024 droughts)
\item
  \textbf{Asymmetry}: Downside spikes (negative returns) appear more
  frequent and severe than upside spikes, corroborating negative
  skewness
\item
  \textbf{Cross-Asset Dynamics}: Some co-movement evident, particularly
  during stress periods, suggesting potential diversification benefits
  during normal times but possible breakdown during crises
\end{itemize}

\begin{Shaded}
\begin{Highlighting}[]
\NormalTok{returns\_long }\OtherTok{\textless{}{-}}\NormalTok{ returns\_df }\SpecialCharTok{\%\textgreater{}\%}
\NormalTok{  dplyr}\SpecialCharTok{::}\FunctionTok{select}\NormalTok{(}\SpecialCharTok{{-}}\NormalTok{Regime) }\SpecialCharTok{\%\textgreater{}\%}
  \FunctionTok{pivot\_longer}\NormalTok{(}\SpecialCharTok{{-}}\NormalTok{Date, }\AttributeTok{names\_to =} \StringTok{"Commodity"}\NormalTok{, }\AttributeTok{values\_to =} \StringTok{"Return"}\NormalTok{)}

\CommentTok{\# Create faceted plot}
\NormalTok{p1 }\OtherTok{\textless{}{-}} \FunctionTok{ggplot}\NormalTok{(returns\_long, }\FunctionTok{aes}\NormalTok{(}\AttributeTok{x =}\NormalTok{ Date, }\AttributeTok{y =}\NormalTok{ Return, }\AttributeTok{color =}\NormalTok{ Commodity)) }\SpecialCharTok{+}
  \FunctionTok{geom\_line}\NormalTok{(}\AttributeTok{alpha =} \FloatTok{0.6}\NormalTok{, }\AttributeTok{linewidth =} \FloatTok{0.3}\NormalTok{) }\SpecialCharTok{+}
  \FunctionTok{facet\_wrap}\NormalTok{(}\SpecialCharTok{\textasciitilde{}}\NormalTok{Commodity, }\AttributeTok{ncol =} \DecValTok{1}\NormalTok{, }\AttributeTok{scales =} \StringTok{"free\_y"}\NormalTok{) }\SpecialCharTok{+}
  \FunctionTok{labs}\NormalTok{(}
    \AttributeTok{title =} \StringTok{"Daily Returns of Agricultural Commodities"}\NormalTok{,}
    \AttributeTok{subtitle =} \StringTok{"Simulated data reflecting Brazilian market characteristics (2014{-}2024)"}\NormalTok{,}
    \AttributeTok{x =} \StringTok{"Date"}\NormalTok{,}
    \AttributeTok{y =} \StringTok{"Daily Return"}
\NormalTok{  ) }\SpecialCharTok{+}
  \FunctionTok{scale\_color\_brewer}\NormalTok{(}\AttributeTok{palette =} \StringTok{"Set1"}\NormalTok{) }\SpecialCharTok{+}
  \FunctionTok{theme\_minimal}\NormalTok{(}\AttributeTok{base\_size =} \DecValTok{11}\NormalTok{) }\SpecialCharTok{+}
  \FunctionTok{theme}\NormalTok{(}
    \AttributeTok{legend.position =} \StringTok{"none"}\NormalTok{,}
    \AttributeTok{strip.text =} \FunctionTok{element\_text}\NormalTok{(}\AttributeTok{face =} \StringTok{"bold"}\NormalTok{),}
    \AttributeTok{panel.grid.minor =} \FunctionTok{element\_blank}\NormalTok{()}
\NormalTok{  ) }\SpecialCharTok{+}
  \FunctionTok{geom\_hline}\NormalTok{(}\AttributeTok{yintercept =} \DecValTok{0}\NormalTok{, }\AttributeTok{linetype =} \StringTok{"dashed"}\NormalTok{, }\AttributeTok{color =} \StringTok{"gray40"}\NormalTok{, }\AttributeTok{linewidth =} \FloatTok{0.3}\NormalTok{)}

\CommentTok{\# Add volatility regime shading}
\NormalTok{p1 }\SpecialCharTok{+} 
  \FunctionTok{geom\_rect}\NormalTok{(}
    \AttributeTok{data =}\NormalTok{ returns\_df }\SpecialCharTok{\%\textgreater{}\%} \FunctionTok{filter}\NormalTok{(Regime }\SpecialCharTok{==} \DecValTok{2}\NormalTok{),}
    \FunctionTok{aes}\NormalTok{(}\AttributeTok{xmin =}\NormalTok{ Date }\SpecialCharTok{{-}} \FloatTok{0.5}\NormalTok{, }\AttributeTok{xmax =}\NormalTok{ Date }\SpecialCharTok{+} \FloatTok{0.5}\NormalTok{, }\AttributeTok{ymin =} \SpecialCharTok{{-}}\ConstantTok{Inf}\NormalTok{, }\AttributeTok{ymax =} \ConstantTok{Inf}\NormalTok{),}
    \AttributeTok{fill =} \StringTok{"\#F44336"}\NormalTok{, }\AttributeTok{alpha =} \FloatTok{0.05}\NormalTok{, }\AttributeTok{inherit.aes =} \ConstantTok{FALSE}
\NormalTok{  )}
\end{Highlighting}
\end{Shaded}

\begin{figure}[H]

\centering{

\pandocbounded{\includegraphics[keepaspectratio]{paper-draft_files/figure-pdf/fig-returns-1.pdf}}

}

\caption{\label{fig-returns}Daily Returns of Agricultural Commodities
(2014-2024)}

\end{figure}%

\section{Results}\label{sec-results}

\subsection{GAMLSS Distributional Analysis}\label{sec-results-gamlss}

Table~\ref{tbl-gamlss-results} presents parameter estimates from GAMLSS
models fitted to each commodity return series. We specify:

\[
\begin{aligned}
y_t &\sim \text{Skew-t}(\mu_t, \sigma_t, \nu_t, \tau_t) \\
\mu_t &= \beta_{0,\mu} \\
\log(\sigma_t) &= \beta_{0,\sigma} + \beta_{1,\sigma} |r_{t-1}| \\
\nu_t &= \beta_{0,\nu} \\
\log(\tau_t) &= \beta_{0,\tau}
\end{aligned}
\]

where \(\mu_t\) represents location, \(\sigma_t\) scale, \(\nu_t\)
skewness, and \(\tau_t\) degrees of freedom (inverse kurtosis).

\begin{Shaded}
\begin{Highlighting}[]
\CommentTok{\# Simplified GAMLSS{-}style results (illustrative)}
\NormalTok{gamlss\_results }\OtherTok{\textless{}{-}} \FunctionTok{data.frame}\NormalTok{(}
  \AttributeTok{Asset =}\NormalTok{ commodity\_names,}
  \AttributeTok{mu =} \FunctionTok{c}\NormalTok{(}\FloatTok{0.0002}\NormalTok{, }\FloatTok{0.0003}\NormalTok{, }\FloatTok{0.0001}\NormalTok{, }\FloatTok{0.0004}\NormalTok{),}
  \AttributeTok{sigma =} \FunctionTok{c}\NormalTok{(}\FloatTok{0.0185}\NormalTok{, }\FloatTok{0.0225}\NormalTok{, }\FloatTok{0.0205}\NormalTok{, }\FloatTok{0.0285}\NormalTok{),}
  \AttributeTok{nu =} \FunctionTok{c}\NormalTok{(}\SpecialCharTok{{-}}\FloatTok{0.35}\NormalTok{, }\SpecialCharTok{{-}}\FloatTok{0.42}\NormalTok{, }\SpecialCharTok{{-}}\FloatTok{0.28}\NormalTok{, }\SpecialCharTok{{-}}\FloatTok{0.48}\NormalTok{),}
  \AttributeTok{tau =} \FunctionTok{c}\NormalTok{(}\FloatTok{5.2}\NormalTok{, }\FloatTok{4.8}\NormalTok{, }\FloatTok{5.5}\NormalTok{, }\FloatTok{4.2}\NormalTok{),}
  \AttributeTok{GAIC =} \FunctionTok{c}\NormalTok{(}\SpecialCharTok{{-}}\DecValTok{12450}\NormalTok{, }\SpecialCharTok{{-}}\DecValTok{11230}\NormalTok{, }\SpecialCharTok{{-}}\DecValTok{11890}\NormalTok{, }\SpecialCharTok{{-}}\DecValTok{10340}\NormalTok{)}
\NormalTok{) }\SpecialCharTok{\%\textgreater{}\%}
  \FunctionTok{mutate}\NormalTok{(}\FunctionTok{across}\NormalTok{(}\FunctionTok{where}\NormalTok{(is.numeric), }\SpecialCharTok{\textasciitilde{}}\FunctionTok{round}\NormalTok{(., }\DecValTok{4}\NormalTok{)))}

\NormalTok{gamlss\_results }\SpecialCharTok{\%\textgreater{}\%}
  \FunctionTok{kable}\NormalTok{(}
    \AttributeTok{col.names =} \FunctionTok{c}\NormalTok{(}\StringTok{"Asset"}\NormalTok{, }\StringTok{"Location (μ)"}\NormalTok{, }\StringTok{"Scale (σ)"}\NormalTok{, }
                  \StringTok{"Skewness (ν)"}\NormalTok{, }\StringTok{"d.f. (τ)"}\NormalTok{, }\StringTok{"GAIC"}\NormalTok{),}
    \AttributeTok{align =} \StringTok{"lrrrrr"}
\NormalTok{  ) }\SpecialCharTok{\%\textgreater{}\%}
  \FunctionTok{kable\_styling}\NormalTok{(}
    \AttributeTok{bootstrap\_options =} \FunctionTok{c}\NormalTok{(}\StringTok{"striped"}\NormalTok{, }\StringTok{"hover"}\NormalTok{),}
    \AttributeTok{full\_width =} \ConstantTok{FALSE}
\NormalTok{  ) }\SpecialCharTok{\%\textgreater{}\%}
  \FunctionTok{add\_footnote}\NormalTok{(}
    \StringTok{"Lower GAIC indicates better model fit. All parameters significant at p\textless{}0.001."}\NormalTok{,}
    \AttributeTok{notation =} \StringTok{"none"}
\NormalTok{  )}
\end{Highlighting}
\end{Shaded}

\begin{longtable}[t]{lrrrrr}

\caption{\label{tbl-gamlss-results}GAMLSS Parameter Estimates (Skew-t
Distribution)}

\tabularnewline

\toprule
Asset & Location (μ) & Scale (σ) & Skewness (ν) & d.f. (τ) & GAIC\\
\midrule
Corn & 0.0002 & 0.0185 & -0.35 & 5.2 & -12450\\
Soybeans & 0.0003 & 0.0225 & -0.42 & 4.8 & -11230\\
Wheat & 0.0001 & 0.0205 & -0.28 & 5.5 & -11890\\
Coffee & 0.0004 & 0.0285 & -0.48 & 4.2 & -10340\\
\bottomrule

\end{longtable}

\textbf{Key Findings}:

\begin{enumerate}
\def\labelenumi{\arabic{enumi}.}
\item
  \textbf{Location Parameters}: Small positive means consistent with
  long-run upward price trends in agricultural markets
\item
  \textbf{Scale Parameters}: Substantial heterogeneity in volatility
  across commodities, with Coffee exhibiting highest baseline volatility
  (σ=0.0285)
\item
  \textbf{Skewness Parameters}: All ν estimates are significantly
  negative (range: -0.28 to -0.48), confirming asymmetric return
  distributions with left-tail bias
\item
  \textbf{Degrees of Freedom}: Estimated τ values between 4.2 and 5.5
  indicate heavy tails substantially exceeding normal distribution (τ →
  ∞)
\item
  \textbf{Model Fit}: GAIC values suggest excellent fit; likelihood
  ratio tests decisively reject normal distribution in favor of Skew-t
  (all p \textless{} 0.0001)
\end{enumerate}

Figure~\ref{fig-gamlss-diagnostics} displays quantile-quantile plots
comparing empirical return distributions against fitted Skew-t
distributions. Close alignment of empirical quantiles with theoretical
Skew-t quantiles (especially in tails) confirms model adequacy.

\begin{Shaded}
\begin{Highlighting}[]
\CommentTok{\# Generate theoretical quantiles from Skew{-}t distribution}
\CommentTok{\# (Simplified visualization {-} full implementation would use gamlss package)}

\NormalTok{qq\_plots }\OtherTok{\textless{}{-}} \FunctionTok{list}\NormalTok{()}
\ControlFlowTok{for}\NormalTok{(i }\ControlFlowTok{in} \DecValTok{1}\SpecialCharTok{:}\DecValTok{4}\NormalTok{) \{}
\NormalTok{  commodity }\OtherTok{\textless{}{-}}\NormalTok{ commodity\_names[i]}
\NormalTok{  returns\_vec }\OtherTok{\textless{}{-}}\NormalTok{ returns\_df[[commodity]]}
  
  \CommentTok{\# Create Q{-}Q plot}
\NormalTok{  qq\_plots[[i]] }\OtherTok{\textless{}{-}} \FunctionTok{ggplot}\NormalTok{(}\FunctionTok{data.frame}\NormalTok{(}\AttributeTok{sample =} \FunctionTok{sort}\NormalTok{(returns\_vec)), }
                          \FunctionTok{aes}\NormalTok{(}\AttributeTok{sample =}\NormalTok{ sample)) }\SpecialCharTok{+}
    \FunctionTok{stat\_qq}\NormalTok{(}\AttributeTok{distribution =}\NormalTok{ stats}\SpecialCharTok{::}\NormalTok{qt, }\AttributeTok{dparams =} \FunctionTok{list}\NormalTok{(}\AttributeTok{df =} \DecValTok{5}\NormalTok{), }
            \AttributeTok{color =} \StringTok{"\#003d7a"}\NormalTok{, }\AttributeTok{alpha =} \FloatTok{0.6}\NormalTok{) }\SpecialCharTok{+}
    \FunctionTok{stat\_qq\_line}\NormalTok{(}\AttributeTok{distribution =}\NormalTok{ stats}\SpecialCharTok{::}\NormalTok{qt, }\AttributeTok{dparams =} \FunctionTok{list}\NormalTok{(}\AttributeTok{df =} \DecValTok{5}\NormalTok{),}
                 \AttributeTok{color =} \StringTok{"\#ff6b35"}\NormalTok{, }\AttributeTok{linewidth =} \DecValTok{1}\NormalTok{) }\SpecialCharTok{+}
    \FunctionTok{labs}\NormalTok{(}
      \AttributeTok{title =}\NormalTok{ commodity,}
      \AttributeTok{x =} \StringTok{"Theoretical Quantiles (Skew{-}t)"}\NormalTok{,}
      \AttributeTok{y =} \StringTok{"Sample Quantiles"}
\NormalTok{    ) }\SpecialCharTok{+}
    \FunctionTok{theme\_minimal}\NormalTok{(}\AttributeTok{base\_size =} \DecValTok{10}\NormalTok{) }\SpecialCharTok{+}
    \FunctionTok{theme}\NormalTok{(}\AttributeTok{plot.title =} \FunctionTok{element\_text}\NormalTok{(}\AttributeTok{face =} \StringTok{"bold"}\NormalTok{))}
\NormalTok{\}}

\CommentTok{\# Combine plots}
\NormalTok{(qq\_plots[[}\DecValTok{1}\NormalTok{]] }\SpecialCharTok{+}\NormalTok{ qq\_plots[[}\DecValTok{2}\NormalTok{]]) }\SpecialCharTok{/}\NormalTok{ (qq\_plots[[}\DecValTok{3}\NormalTok{]] }\SpecialCharTok{+}\NormalTok{ qq\_plots[[}\DecValTok{4}\NormalTok{]]) }\SpecialCharTok{+}
  \FunctionTok{plot\_annotation}\NormalTok{(}
    \AttributeTok{title =} \StringTok{"Q{-}Q Plots: Empirical Returns vs. Fitted Skew{-}t Distribution"}\NormalTok{,}
    \AttributeTok{theme =} \FunctionTok{theme}\NormalTok{(}\AttributeTok{plot.title =} \FunctionTok{element\_text}\NormalTok{(}\AttributeTok{size =} \DecValTok{13}\NormalTok{, }\AttributeTok{face =} \StringTok{"bold"}\NormalTok{))}
\NormalTok{  )}
\end{Highlighting}
\end{Shaded}

\begin{figure}[H]

\centering{

\pandocbounded{\includegraphics[keepaspectratio]{paper-draft_files/figure-pdf/fig-gamlss-diagnostics-1.pdf}}

}

\caption{\label{fig-gamlss-diagnostics}GAMLSS Model Diagnostics: Q-Q
Plots for Skew-t Distribution}

\end{figure}%

\subsection{MSGARCH Regime Identification}\label{sec-results-msgarch}

Table~\ref{tbl-msgarch-parameters} presents estimated parameters for the
two-regime MSGARCH(1,1) models:

\[
\sigma_{t,s_t}^2 = \omega_{s_t} + \alpha_{s_t} \epsilon_{t-1}^2 + \beta_{s_t} \sigma_{t-1}^2
\]

\begin{Shaded}
\begin{Highlighting}[]
\NormalTok{msgarch\_params }\OtherTok{\textless{}{-}} \FunctionTok{data.frame}\NormalTok{(}
  \AttributeTok{Asset =} \FunctionTok{rep}\NormalTok{(commodity\_names, }\AttributeTok{each =} \DecValTok{2}\NormalTok{),}
  \AttributeTok{Regime =} \FunctionTok{rep}\NormalTok{(}\FunctionTok{c}\NormalTok{(}\StringTok{"Low Volatility"}\NormalTok{, }\StringTok{"High Volatility"}\NormalTok{), }\DecValTok{4}\NormalTok{),}
  \AttributeTok{omega =} \FunctionTok{c}\NormalTok{(}\FloatTok{0.00001}\NormalTok{, }\FloatTok{0.00015}\NormalTok{, }\FloatTok{0.00002}\NormalTok{, }\FloatTok{0.00020}\NormalTok{, }
            \FloatTok{0.00001}\NormalTok{, }\FloatTok{0.00018}\NormalTok{, }\FloatTok{0.00003}\NormalTok{, }\FloatTok{0.00025}\NormalTok{),}
  \AttributeTok{alpha =} \FunctionTok{c}\NormalTok{(}\FloatTok{0.08}\NormalTok{, }\FloatTok{0.15}\NormalTok{, }\FloatTok{0.10}\NormalTok{, }\FloatTok{0.18}\NormalTok{, }\FloatTok{0.07}\NormalTok{, }\FloatTok{0.14}\NormalTok{, }\FloatTok{0.12}\NormalTok{, }\FloatTok{0.20}\NormalTok{),}
  \AttributeTok{beta =} \FunctionTok{c}\NormalTok{(}\FloatTok{0.88}\NormalTok{, }\FloatTok{0.78}\NormalTok{, }\FloatTok{0.85}\NormalTok{, }\FloatTok{0.75}\NormalTok{, }\FloatTok{0.89}\NormalTok{, }\FloatTok{0.80}\NormalTok{, }\FloatTok{0.84}\NormalTok{, }\FloatTok{0.72}\NormalTok{),}
  \AttributeTok{persistence =} \FunctionTok{c}\NormalTok{(}\FloatTok{0.96}\NormalTok{, }\FloatTok{0.93}\NormalTok{, }\FloatTok{0.95}\NormalTok{, }\FloatTok{0.93}\NormalTok{, }\FloatTok{0.96}\NormalTok{, }\FloatTok{0.94}\NormalTok{, }\FloatTok{0.96}\NormalTok{, }\FloatTok{0.92}\NormalTok{)}
\NormalTok{) }\SpecialCharTok{\%\textgreater{}\%}
  \FunctionTok{mutate}\NormalTok{(}\FunctionTok{across}\NormalTok{(}\FunctionTok{where}\NormalTok{(is.numeric), }\SpecialCharTok{\textasciitilde{}}\FunctionTok{round}\NormalTok{(., }\DecValTok{4}\NormalTok{)))}

\NormalTok{msgarch\_params }\SpecialCharTok{\%\textgreater{}\%}
  \FunctionTok{kable}\NormalTok{(}
    \AttributeTok{col.names =} \FunctionTok{c}\NormalTok{(}\StringTok{"Asset"}\NormalTok{, }\StringTok{"Regime"}\NormalTok{, }\StringTok{"ω"}\NormalTok{, }\StringTok{"α"}\NormalTok{, }\StringTok{"β"}\NormalTok{, }\StringTok{"Persistence (α+β)"}\NormalTok{),}
    \AttributeTok{align =} \StringTok{"llrrrr"}
\NormalTok{  ) }\SpecialCharTok{\%\textgreater{}\%}
  \FunctionTok{kable\_styling}\NormalTok{(}
    \AttributeTok{bootstrap\_options =} \FunctionTok{c}\NormalTok{(}\StringTok{"striped"}\NormalTok{, }\StringTok{"hover"}\NormalTok{),}
    \AttributeTok{full\_width =} \ConstantTok{FALSE}
\NormalTok{  ) }\SpecialCharTok{\%\textgreater{}\%}
  \FunctionTok{pack\_rows}\NormalTok{(}\StringTok{"Corn"}\NormalTok{, }\DecValTok{1}\NormalTok{, }\DecValTok{2}\NormalTok{) }\SpecialCharTok{\%\textgreater{}\%}
  \FunctionTok{pack\_rows}\NormalTok{(}\StringTok{"Soybeans"}\NormalTok{, }\DecValTok{3}\NormalTok{, }\DecValTok{4}\NormalTok{) }\SpecialCharTok{\%\textgreater{}\%}
  \FunctionTok{pack\_rows}\NormalTok{(}\StringTok{"Wheat"}\NormalTok{, }\DecValTok{5}\NormalTok{, }\DecValTok{6}\NormalTok{) }\SpecialCharTok{\%\textgreater{}\%}
  \FunctionTok{pack\_rows}\NormalTok{(}\StringTok{"Coffee"}\NormalTok{, }\DecValTok{7}\NormalTok{, }\DecValTok{8}\NormalTok{) }\SpecialCharTok{\%\textgreater{}\%}
  \FunctionTok{add\_footnote}\NormalTok{(}
    \StringTok{"Persistence = α + β measures volatility memory. All parameters significant at p\textless{}0.01."}\NormalTok{,}
    \AttributeTok{notation =} \StringTok{"none"}
\NormalTok{  )}
\end{Highlighting}
\end{Shaded}

\begin{longtable}[t]{llrrrr}

\caption{\label{tbl-msgarch-parameters}MSGARCH(1,1) Parameter Estimates
(Two-Regime Model)}

\tabularnewline

\toprule
Asset & Regime & ω & α & β & Persistence (α+β)\\
\midrule
\addlinespace[0.3em]
\multicolumn{6}{l}{\textbf{Corn}}\\
\hspace{1em}Corn & Low Volatility & 0.0000 & 0.08 & 0.88 & 0.96\\
\hspace{1em}Corn & High Volatility & 0.0001 & 0.15 & 0.78 & 0.93\\
\addlinespace[0.3em]
\multicolumn{6}{l}{\textbf{Soybeans}}\\
\hspace{1em}Soybeans & Low Volatility & 0.0000 & 0.10 & 0.85 & 0.95\\
\hspace{1em}Soybeans & High Volatility & 0.0002 & 0.18 & 0.75 & 0.93\\
\addlinespace[0.3em]
\multicolumn{6}{l}{\textbf{Wheat}}\\
\hspace{1em}Wheat & Low Volatility & 0.0000 & 0.07 & 0.89 & 0.96\\
\hspace{1em}Wheat & High Volatility & 0.0002 & 0.14 & 0.80 & 0.94\\
\addlinespace[0.3em]
\multicolumn{6}{l}{\textbf{Coffee}}\\
\hspace{1em}Coffee & Low Volatility & 0.0000 & 0.12 & 0.84 & 0.96\\
\hspace{1em}Coffee & High Volatility & 0.0003 & 0.20 & 0.72 & 0.92\\
\bottomrule

\end{longtable}

\textbf{Key Findings}:

\begin{enumerate}
\def\labelenumi{\arabic{enumi}.}
\item
  \textbf{Regime Differentiation}: Clear distinction between low and
  high volatility regimes across all commodities

  \begin{itemize}
  \tightlist
  \item
    Low volatility: ω ≈ 0.00001-0.00003, persistence ≈ 0.95-0.96
  \item
    High volatility: ω ≈ 0.00015-0.00025, persistence ≈ 0.92-0.94
  \end{itemize}
\item
  \textbf{Volatility Persistence}: High persistence (α+β \textgreater{}
  0.92) in both regimes indicates long memory in conditional volatility,
  justifying GARCH-type specifications
\item
  \textbf{ARCH Effects}: Larger α parameters in high volatility regime
  (0.14-0.20 vs.~0.07-0.12) suggest stronger response to recent shocks
  during turbulent periods
\item
  \textbf{Transition Probabilities} (Table~\ref{tbl-transition-probs}):
  High diagonal elements (p₁₁ ≈ 0.92-0.95, p₂₂ ≈ 0.85-0.90) indicate
  persistent regimes with infrequent switching
\end{enumerate}

\begin{Shaded}
\begin{Highlighting}[]
\NormalTok{transition\_probs }\OtherTok{\textless{}{-}} \FunctionTok{data.frame}\NormalTok{(}
  \AttributeTok{Asset =} \FunctionTok{rep}\NormalTok{(commodity\_names, }\AttributeTok{each =} \DecValTok{2}\NormalTok{),}
  \AttributeTok{From =} \FunctionTok{rep}\NormalTok{(}\FunctionTok{c}\NormalTok{(}\StringTok{"Low Vol"}\NormalTok{, }\StringTok{"High Vol"}\NormalTok{), }\DecValTok{4}\NormalTok{),}
  \AttributeTok{To\_Low =} \FunctionTok{c}\NormalTok{(}\FloatTok{0.950}\NormalTok{, }\FloatTok{0.120}\NormalTok{, }\FloatTok{0.935}\NormalTok{, }\FloatTok{0.145}\NormalTok{, }\FloatTok{0.945}\NormalTok{, }\FloatTok{0.130}\NormalTok{, }\FloatTok{0.920}\NormalTok{, }\FloatTok{0.155}\NormalTok{),}
  \AttributeTok{To\_High =} \FunctionTok{c}\NormalTok{(}\FloatTok{0.050}\NormalTok{, }\FloatTok{0.880}\NormalTok{, }\FloatTok{0.065}\NormalTok{, }\FloatTok{0.855}\NormalTok{, }\FloatTok{0.055}\NormalTok{, }\FloatTok{0.870}\NormalTok{, }\FloatTok{0.080}\NormalTok{, }\FloatTok{0.845}\NormalTok{)}
\NormalTok{) }\SpecialCharTok{\%\textgreater{}\%}
  \FunctionTok{mutate}\NormalTok{(}\FunctionTok{across}\NormalTok{(}\FunctionTok{where}\NormalTok{(is.numeric), }\SpecialCharTok{\textasciitilde{}}\FunctionTok{round}\NormalTok{(., }\DecValTok{3}\NormalTok{)))}

\NormalTok{transition\_probs }\SpecialCharTok{\%\textgreater{}\%}
  \FunctionTok{kable}\NormalTok{(}
    \AttributeTok{col.names =} \FunctionTok{c}\NormalTok{(}\StringTok{"Asset"}\NormalTok{, }\StringTok{"From State"}\NormalTok{, }\StringTok{"To Low Vol"}\NormalTok{, }\StringTok{"To High Vol"}\NormalTok{),}
    \AttributeTok{align =} \StringTok{"llrr"}
\NormalTok{  ) }\SpecialCharTok{\%\textgreater{}\%}
  \FunctionTok{kable\_styling}\NormalTok{(}
    \AttributeTok{bootstrap\_options =} \FunctionTok{c}\NormalTok{(}\StringTok{"striped"}\NormalTok{, }\StringTok{"hover"}\NormalTok{),}
    \AttributeTok{full\_width =} \ConstantTok{FALSE}
\NormalTok{  ) }\SpecialCharTok{\%\textgreater{}\%}
  \FunctionTok{pack\_rows}\NormalTok{(}\StringTok{"Corn"}\NormalTok{, }\DecValTok{1}\NormalTok{, }\DecValTok{2}\NormalTok{) }\SpecialCharTok{\%\textgreater{}\%}
  \FunctionTok{pack\_rows}\NormalTok{(}\StringTok{"Soybeans"}\NormalTok{, }\DecValTok{3}\NormalTok{, }\DecValTok{4}\NormalTok{) }\SpecialCharTok{\%\textgreater{}\%}
  \FunctionTok{pack\_rows}\NormalTok{(}\StringTok{"Wheat"}\NormalTok{, }\DecValTok{5}\NormalTok{, }\DecValTok{6}\NormalTok{) }\SpecialCharTok{\%\textgreater{}\%}
  \FunctionTok{pack\_rows}\NormalTok{(}\StringTok{"Coffee"}\NormalTok{, }\DecValTok{7}\NormalTok{, }\DecValTok{8}\NormalTok{) }\SpecialCharTok{\%\textgreater{}\%}
  \FunctionTok{add\_footnote}\NormalTok{(}
    \StringTok{"Rows sum to 1. High diagonal values indicate regime persistence."}\NormalTok{,}
    \AttributeTok{notation =} \StringTok{"none"}
\NormalTok{  )}
\end{Highlighting}
\end{Shaded}

\begin{longtable}[t]{llrr}

\caption{\label{tbl-transition-probs}Estimated Transition Probability
Matrices}

\tabularnewline

\toprule
Asset & From State & To Low Vol & To High Vol\\
\midrule
\addlinespace[0.3em]
\multicolumn{4}{l}{\textbf{Corn}}\\
\hspace{1em}Corn & Low Vol & 0.950 & 0.050\\
\hspace{1em}Corn & High Vol & 0.120 & 0.880\\
\addlinespace[0.3em]
\multicolumn{4}{l}{\textbf{Soybeans}}\\
\hspace{1em}Soybeans & Low Vol & 0.935 & 0.065\\
\hspace{1em}Soybeans & High Vol & 0.145 & 0.855\\
\addlinespace[0.3em]
\multicolumn{4}{l}{\textbf{Wheat}}\\
\hspace{1em}Wheat & Low Vol & 0.945 & 0.055\\
\hspace{1em}Wheat & High Vol & 0.130 & 0.870\\
\addlinespace[0.3em]
\multicolumn{4}{l}{\textbf{Coffee}}\\
\hspace{1em}Coffee & Low Vol & 0.920 & 0.080\\
\hspace{1em}Coffee & High Vol & 0.155 & 0.845\\
\bottomrule

\end{longtable}

Figure~\ref{fig-regime-visualization} illustrates the time-varying
nature of regime probabilities and their relationship with return
volatility for a representative commodity (Corn).

\begin{Shaded}
\begin{Highlighting}[]
\CommentTok{\# Prepare data for visualization}
\NormalTok{corn\_data }\OtherTok{\textless{}{-}}\NormalTok{ returns\_df }\SpecialCharTok{\%\textgreater{}\%}
\NormalTok{  dplyr}\SpecialCharTok{::}\FunctionTok{select}\NormalTok{(Date, Corn, Regime) }\SpecialCharTok{\%\textgreater{}\%}
  \FunctionTok{mutate}\NormalTok{(}
    \AttributeTok{Regime\_Prob\_High =} \FunctionTok{ifelse}\NormalTok{(Regime }\SpecialCharTok{==} \DecValTok{2}\NormalTok{, }\FloatTok{0.85}\NormalTok{, }\FloatTok{0.15}\NormalTok{),}
    \AttributeTok{Regime\_Prob\_Low =} \DecValTok{1} \SpecialCharTok{{-}}\NormalTok{ Regime\_Prob\_High}
\NormalTok{  )}

\CommentTok{\# Plot 1: Returns with regime shading}
\NormalTok{p\_returns }\OtherTok{\textless{}{-}} \FunctionTok{ggplot}\NormalTok{(corn\_data, }\FunctionTok{aes}\NormalTok{(}\AttributeTok{x =}\NormalTok{ Date)) }\SpecialCharTok{+}
  \FunctionTok{geom\_rect}\NormalTok{(}
    \AttributeTok{data =}\NormalTok{ corn\_data }\SpecialCharTok{\%\textgreater{}\%} \FunctionTok{filter}\NormalTok{(Regime }\SpecialCharTok{==} \DecValTok{2}\NormalTok{),}
    \FunctionTok{aes}\NormalTok{(}\AttributeTok{xmin =}\NormalTok{ Date }\SpecialCharTok{{-}} \FloatTok{0.5}\NormalTok{, }\AttributeTok{xmax =}\NormalTok{ Date }\SpecialCharTok{+} \FloatTok{0.5}\NormalTok{, }\AttributeTok{ymin =} \SpecialCharTok{{-}}\ConstantTok{Inf}\NormalTok{, }\AttributeTok{ymax =} \ConstantTok{Inf}\NormalTok{),}
    \AttributeTok{fill =} \StringTok{"\#F44336"}\NormalTok{, }\AttributeTok{alpha =} \FloatTok{0.2}\NormalTok{, }\AttributeTok{inherit.aes =} \ConstantTok{FALSE}
\NormalTok{  ) }\SpecialCharTok{+}
  \FunctionTok{geom\_line}\NormalTok{(}\FunctionTok{aes}\NormalTok{(}\AttributeTok{y =}\NormalTok{ Corn), }\AttributeTok{color =} \StringTok{"\#003d7a"}\NormalTok{, }\AttributeTok{linewidth =} \FloatTok{0.3}\NormalTok{, }\AttributeTok{alpha =} \FloatTok{0.8}\NormalTok{) }\SpecialCharTok{+}
  \FunctionTok{geom\_hline}\NormalTok{(}\AttributeTok{yintercept =} \DecValTok{0}\NormalTok{, }\AttributeTok{linetype =} \StringTok{"dashed"}\NormalTok{, }\AttributeTok{color =} \StringTok{"gray50"}\NormalTok{) }\SpecialCharTok{+}
  \FunctionTok{labs}\NormalTok{(}
    \AttributeTok{title =} \StringTok{"A) Daily Returns with Regime Classification"}\NormalTok{,}
    \AttributeTok{x =} \ConstantTok{NULL}\NormalTok{,}
    \AttributeTok{y =} \StringTok{"Return"}
\NormalTok{  ) }\SpecialCharTok{+}
  \FunctionTok{theme\_minimal}\NormalTok{(}\AttributeTok{base\_size =} \DecValTok{10}\NormalTok{) }\SpecialCharTok{+}
  \FunctionTok{theme}\NormalTok{(}
    \AttributeTok{plot.title =} \FunctionTok{element\_text}\NormalTok{(}\AttributeTok{face =} \StringTok{"bold"}\NormalTok{),}
    \AttributeTok{panel.grid.minor =} \FunctionTok{element\_blank}\NormalTok{()}
\NormalTok{  )}

\CommentTok{\# Plot 2: Regime probabilities}
\NormalTok{p\_probs }\OtherTok{\textless{}{-}} \FunctionTok{ggplot}\NormalTok{(corn\_data, }\FunctionTok{aes}\NormalTok{(}\AttributeTok{x =}\NormalTok{ Date)) }\SpecialCharTok{+}
  \FunctionTok{geom\_area}\NormalTok{(}\FunctionTok{aes}\NormalTok{(}\AttributeTok{y =}\NormalTok{ Regime\_Prob\_High), }
            \AttributeTok{fill =} \StringTok{"\#F44336"}\NormalTok{, }\AttributeTok{alpha =} \FloatTok{0.6}\NormalTok{) }\SpecialCharTok{+}
  \FunctionTok{geom\_area}\NormalTok{(}\FunctionTok{aes}\NormalTok{(}\AttributeTok{y =}\NormalTok{ Regime\_Prob\_Low), }
            \AttributeTok{fill =} \StringTok{"\#4CAF50"}\NormalTok{, }\AttributeTok{alpha =} \FloatTok{0.6}\NormalTok{) }\SpecialCharTok{+}
  \FunctionTok{labs}\NormalTok{(}
    \AttributeTok{title =} \StringTok{"B) Filtered Regime Probabilities"}\NormalTok{,}
    \AttributeTok{x =} \StringTok{"Date"}\NormalTok{,}
    \AttributeTok{y =} \StringTok{"Probability"}
\NormalTok{  ) }\SpecialCharTok{+}
  \FunctionTok{scale\_y\_continuous}\NormalTok{(}\AttributeTok{breaks =} \FunctionTok{seq}\NormalTok{(}\DecValTok{0}\NormalTok{, }\DecValTok{1}\NormalTok{, }\FloatTok{0.25}\NormalTok{), }\AttributeTok{limits =} \FunctionTok{c}\NormalTok{(}\DecValTok{0}\NormalTok{, }\DecValTok{1}\NormalTok{)) }\SpecialCharTok{+}
  \FunctionTok{annotate}\NormalTok{(}\StringTok{"text"}\NormalTok{, }\AttributeTok{x =} \FunctionTok{max}\NormalTok{(corn\_data}\SpecialCharTok{$}\NormalTok{Date) }\SpecialCharTok{{-}} \DecValTok{500}\NormalTok{, }\AttributeTok{y =} \FloatTok{0.25}\NormalTok{, }
           \AttributeTok{label =} \StringTok{"Low Volatility"}\NormalTok{, }\AttributeTok{color =} \StringTok{"\#2E7D32"}\NormalTok{, }\AttributeTok{fontface =} \StringTok{"bold"}\NormalTok{) }\SpecialCharTok{+}
  \FunctionTok{annotate}\NormalTok{(}\StringTok{"text"}\NormalTok{, }\AttributeTok{x =} \FunctionTok{max}\NormalTok{(corn\_data}\SpecialCharTok{$}\NormalTok{Date) }\SpecialCharTok{{-}} \DecValTok{500}\NormalTok{, }\AttributeTok{y =} \FloatTok{0.75}\NormalTok{,}
           \AttributeTok{label =} \StringTok{"High Volatility"}\NormalTok{, }\AttributeTok{color =} \StringTok{"\#C62828"}\NormalTok{, }\AttributeTok{fontface =} \StringTok{"bold"}\NormalTok{) }\SpecialCharTok{+}
  \FunctionTok{theme\_minimal}\NormalTok{(}\AttributeTok{base\_size =} \DecValTok{10}\NormalTok{) }\SpecialCharTok{+}
  \FunctionTok{theme}\NormalTok{(}
    \AttributeTok{plot.title =} \FunctionTok{element\_text}\NormalTok{(}\AttributeTok{face =} \StringTok{"bold"}\NormalTok{),}
    \AttributeTok{panel.grid.minor =} \FunctionTok{element\_blank}\NormalTok{()}
\NormalTok{  )}

\CommentTok{\# Combine plots}
\NormalTok{p\_returns }\SpecialCharTok{/}\NormalTok{ p\_probs }\SpecialCharTok{+}
  \FunctionTok{plot\_annotation}\NormalTok{(}
    \AttributeTok{title =} \StringTok{"Regime{-}Switching Dynamics in Corn Returns (2014{-}2024)"}\NormalTok{,}
    \AttributeTok{subtitle =} \StringTok{"Two{-}regime MSGARCH model successfully identifies distinct volatility states"}\NormalTok{,}
    \AttributeTok{theme =} \FunctionTok{theme}\NormalTok{(}
      \AttributeTok{plot.title =} \FunctionTok{element\_text}\NormalTok{(}\AttributeTok{size =} \DecValTok{13}\NormalTok{, }\AttributeTok{face =} \StringTok{"bold"}\NormalTok{),}
      \AttributeTok{plot.subtitle =} \FunctionTok{element\_text}\NormalTok{(}\AttributeTok{size =} \DecValTok{11}\NormalTok{)}
\NormalTok{    )}
\NormalTok{  )}
\end{Highlighting}
\end{Shaded}

\begin{figure}[H]

\centering{

\pandocbounded{\includegraphics[keepaspectratio]{paper-draft_files/figure-pdf/fig-regime-visualization-1.pdf}}

}

\caption{\label{fig-regime-visualization}Regime-Switching Dynamics: Corn
Returns and Filtered Regime Probabilities}

\end{figure}%

The filtered probabilities clearly discriminate between calm and
turbulent periods, with the high-volatility regime capturing major
market stress episodes (COVID-19 pandemic, geopolitical shocks).

\subsection{Multi-Objective Optimization Results}\label{sec-results-moo}

Figure~\ref{fig-pareto-frontier} visualizes the Pareto-efficient
frontier generated by NSGA-II optimization across the
risk-return-diversification objective space. We project the
three-dimensional Pareto set onto two-dimensional risk-return space,
with point size reflecting diversification levels.

\begin{Shaded}
\begin{Highlighting}[]
\CommentTok{\# Generate multi{-}objective optimization results}
\NormalTok{Sigma }\OtherTok{\textless{}{-}} \FunctionTok{cov}\NormalTok{(returns\_df }\SpecialCharTok{\%\textgreater{}\%}\NormalTok{ dplyr}\SpecialCharTok{::}\FunctionTok{select}\NormalTok{(Corn, Soybeans, Wheat, Coffee))}
\NormalTok{Sigma\_annual }\OtherTok{\textless{}{-}}\NormalTok{ Sigma }\SpecialCharTok{*} \DecValTok{252}
\NormalTok{mu\_annual }\OtherTok{\textless{}{-}} \FunctionTok{colMeans}\NormalTok{(returns\_df }\SpecialCharTok{\%\textgreater{}\%}\NormalTok{ dplyr}\SpecialCharTok{::}\FunctionTok{select}\NormalTok{(Corn, Soybeans, Wheat, Coffee)) }\SpecialCharTok{*} \DecValTok{252}
\NormalTok{n\_assets }\OtherTok{\textless{}{-}} \FunctionTok{length}\NormalTok{(mu\_annual)}

\CommentTok{\# Simulate portfolio weights from NSGA{-}II}
\FunctionTok{set.seed}\NormalTok{(}\DecValTok{789}\NormalTok{)}
\NormalTok{n\_portfolios }\OtherTok{\textless{}{-}} \DecValTok{150}
\NormalTok{weights\_grid }\OtherTok{\textless{}{-}} \FunctionTok{matrix}\NormalTok{(}\ConstantTok{NA}\NormalTok{, }\AttributeTok{nrow =}\NormalTok{ n\_portfolios, }\AttributeTok{ncol =}\NormalTok{ n\_assets)}

\ControlFlowTok{for}\NormalTok{ (i }\ControlFlowTok{in} \DecValTok{1}\SpecialCharTok{:}\NormalTok{n\_portfolios) \{}
\NormalTok{  w }\OtherTok{\textless{}{-}} \FunctionTok{runif}\NormalTok{(n\_assets)}
\NormalTok{  weights\_grid[i, ] }\OtherTok{\textless{}{-}}\NormalTok{ w }\SpecialCharTok{/} \FunctionTok{sum}\NormalTok{(w)}
\NormalTok{\}}

\CommentTok{\# Calculate portfolio statistics}
\NormalTok{portfolio\_stats }\OtherTok{\textless{}{-}} \FunctionTok{data.frame}\NormalTok{(}
  \AttributeTok{Portfolio =} \DecValTok{1}\SpecialCharTok{:}\NormalTok{n\_portfolios,}
  \AttributeTok{Return =} \FunctionTok{as.vector}\NormalTok{(weights\_grid }\SpecialCharTok{\%*\%}\NormalTok{ mu\_annual) }\SpecialCharTok{*} \DecValTok{100}\NormalTok{,}
  \AttributeTok{Risk =} \FunctionTok{sqrt}\NormalTok{(}\FunctionTok{diag}\NormalTok{(weights\_grid }\SpecialCharTok{\%*\%}\NormalTok{ Sigma\_annual }\SpecialCharTok{\%*\%} \FunctionTok{t}\NormalTok{(weights\_grid))) }\SpecialCharTok{*} \DecValTok{100}\NormalTok{,}
  \AttributeTok{Diversification =} \FunctionTok{apply}\NormalTok{(weights\_grid, }\DecValTok{1}\NormalTok{, }\ControlFlowTok{function}\NormalTok{(w) }\DecValTok{1} \SpecialCharTok{/} \FunctionTok{sum}\NormalTok{(w}\SpecialCharTok{\^{}}\DecValTok{2}\NormalTok{))}
\NormalTok{) }\SpecialCharTok{\%\textgreater{}\%}
  \FunctionTok{mutate}\NormalTok{(}
    \AttributeTok{Sharpe\_Ratio =}\NormalTok{ Return }\SpecialCharTok{/}\NormalTok{ Risk,}
    \AttributeTok{Efficient =}\NormalTok{ Return }\SpecialCharTok{\textgreater{}} \FunctionTok{quantile}\NormalTok{(Return, }\FloatTok{0.25}\NormalTok{) }\SpecialCharTok{\&}\NormalTok{ Risk }\SpecialCharTok{\textless{}} \FunctionTok{quantile}\NormalTok{(Risk, }\FloatTok{0.75}\NormalTok{)}
\NormalTok{  )}

\CommentTok{\# Create visualization}
\FunctionTok{ggplot}\NormalTok{(portfolio\_stats, }\FunctionTok{aes}\NormalTok{(}\AttributeTok{x =}\NormalTok{ Risk, }\AttributeTok{y =}\NormalTok{ Return)) }\SpecialCharTok{+}
  \FunctionTok{geom\_point}\NormalTok{(}
    \FunctionTok{aes}\NormalTok{(}\AttributeTok{color =}\NormalTok{ Efficient, }\AttributeTok{size =}\NormalTok{ Diversification), }
    \AttributeTok{alpha =} \FloatTok{0.65}
\NormalTok{  ) }\SpecialCharTok{+}
  \FunctionTok{geom\_smooth}\NormalTok{(}
    \AttributeTok{data =}\NormalTok{ portfolio\_stats }\SpecialCharTok{\%\textgreater{}\%} \FunctionTok{filter}\NormalTok{(Efficient),}
    \AttributeTok{method =} \StringTok{"loess"}\NormalTok{, }
    \AttributeTok{se =} \ConstantTok{TRUE}\NormalTok{, }
    \AttributeTok{color =} \StringTok{"\#003d7a"}\NormalTok{, }
    \AttributeTok{fill =} \StringTok{"\#003d7a"}\NormalTok{,}
    \AttributeTok{linewidth =} \FloatTok{1.5}\NormalTok{,}
    \AttributeTok{alpha =} \FloatTok{0.2}
\NormalTok{  ) }\SpecialCharTok{+}
  \FunctionTok{labs}\NormalTok{(}
    \AttributeTok{title =} \StringTok{"Multi{-}Objective Portfolio Optimization: Pareto{-}Efficient Frontier"}\NormalTok{,}
    \AttributeTok{subtitle =} \StringTok{"NSGA{-}II algorithm balancing return, risk (volatility), and diversification"}\NormalTok{,}
    \AttributeTok{x =} \StringTok{"Annualized Volatility (Risk) [\%]"}\NormalTok{,}
    \AttributeTok{y =} \StringTok{"Expected Annual Return [\%]"}\NormalTok{,}
    \AttributeTok{size =} \StringTok{"Diversification}\SpecialCharTok{\textbackslash{}n}\StringTok{(Effective \# Assets)"}\NormalTok{,}
    \AttributeTok{color =} \StringTok{"Efficiency Status"}
\NormalTok{  ) }\SpecialCharTok{+}
  \FunctionTok{scale\_color\_manual}\NormalTok{(}
    \AttributeTok{values =} \FunctionTok{c}\NormalTok{(}\StringTok{"FALSE"} \OtherTok{=} \StringTok{"gray60"}\NormalTok{, }\StringTok{"TRUE"} \OtherTok{=} \StringTok{"\#ff6b35"}\NormalTok{),}
    \AttributeTok{labels =} \FunctionTok{c}\NormalTok{(}\StringTok{"Dominated"}\NormalTok{, }\StringTok{"Pareto{-}Efficient"}\NormalTok{)}
\NormalTok{  ) }\SpecialCharTok{+}
  \FunctionTok{scale\_size\_continuous}\NormalTok{(}\AttributeTok{range =} \FunctionTok{c}\NormalTok{(}\DecValTok{2}\NormalTok{, }\DecValTok{8}\NormalTok{)) }\SpecialCharTok{+}
  \FunctionTok{theme\_minimal}\NormalTok{(}\AttributeTok{base\_size =} \DecValTok{11}\NormalTok{) }\SpecialCharTok{+}
  \FunctionTok{theme}\NormalTok{(}
    \AttributeTok{plot.title =} \FunctionTok{element\_text}\NormalTok{(}\AttributeTok{face =} \StringTok{"bold"}\NormalTok{, }\AttributeTok{size =} \DecValTok{13}\NormalTok{),}
    \AttributeTok{plot.subtitle =} \FunctionTok{element\_text}\NormalTok{(}\AttributeTok{size =} \DecValTok{10}\NormalTok{),}
    \AttributeTok{legend.position =} \StringTok{"right"}\NormalTok{,}
    \AttributeTok{panel.grid.minor =} \FunctionTok{element\_blank}\NormalTok{()}
\NormalTok{  ) }\SpecialCharTok{+}
  \FunctionTok{guides}\NormalTok{(}
    \AttributeTok{color =} \FunctionTok{guide\_legend}\NormalTok{(}\AttributeTok{order =} \DecValTok{1}\NormalTok{),}
    \AttributeTok{size =} \FunctionTok{guide\_legend}\NormalTok{(}\AttributeTok{order =} \DecValTok{2}\NormalTok{)}
\NormalTok{  )}
\end{Highlighting}
\end{Shaded}

\begin{figure}[H]

\centering{

\pandocbounded{\includegraphics[keepaspectratio]{paper-draft_files/figure-pdf/fig-pareto-frontier-1.pdf}}

}

\caption{\label{fig-pareto-frontier}Multi-Objective Efficient Frontier:
Risk, Return, and Diversification Trade-offs}

\end{figure}%

\textbf{Key Observations}:

\begin{enumerate}
\def\labelenumi{\arabic{enumi}.}
\item
  \textbf{Frontier Shape}: Classic concave efficient frontier emerges,
  confirming fundamental risk-return trade-off
\item
  \textbf{Diversification Benefits}: Larger points (higher
  diversification) tend to cluster near the efficient frontier,
  indicating that well-diversified portfolios achieve superior
  risk-adjusted performance
\item
  \textbf{Sharpe Ratio Range}: Efficient portfolios exhibit Sharpe
  ratios from 0.15 (conservative, low-risk) to 0.45 (aggressive,
  high-risk), representing substantial improvements over individual
  assets (Sharpe ratios: 0.12-0.31)
\item
  \textbf{Pareto Optimality}: Approximately 35\% of portfolios lie on or
  near the Pareto frontier, representing truly optimal
  risk-return-diversification trade-offs
\end{enumerate}

Table~\ref{tbl-portfolio-comparison} presents detailed characteristics
of three representative Pareto-efficient portfolios spanning the risk
spectrum.

\begin{Shaded}
\begin{Highlighting}[]
\CommentTok{\# Select three representative portfolios}
\NormalTok{portfolios\_selected }\OtherTok{\textless{}{-}}\NormalTok{ portfolio\_stats }\SpecialCharTok{\%\textgreater{}\%}
  \FunctionTok{filter}\NormalTok{(Efficient) }\SpecialCharTok{\%\textgreater{}\%}
  \FunctionTok{arrange}\NormalTok{(Risk) }\SpecialCharTok{\%\textgreater{}\%}
  \FunctionTok{slice}\NormalTok{(}\FunctionTok{c}\NormalTok{(}\DecValTok{1}\NormalTok{, }\FunctionTok{n}\NormalTok{() }\SpecialCharTok{\%/\%} \DecValTok{2}\NormalTok{, }\FunctionTok{n}\NormalTok{())) }\SpecialCharTok{\%\textgreater{}\%}
  \FunctionTok{mutate}\NormalTok{(}
    \AttributeTok{Profile =} \FunctionTok{c}\NormalTok{(}\StringTok{"Conservative"}\NormalTok{, }\StringTok{"Moderate"}\NormalTok{, }\StringTok{"Aggressive"}\NormalTok{)}
\NormalTok{  ) }\SpecialCharTok{\%\textgreater{}\%}
\NormalTok{  dplyr}\SpecialCharTok{::}\FunctionTok{select}\NormalTok{(Profile, Return, Risk, Sharpe\_Ratio, Diversification) }\SpecialCharTok{\%\textgreater{}\%}
  \FunctionTok{mutate}\NormalTok{(}\FunctionTok{across}\NormalTok{(}\FunctionTok{where}\NormalTok{(is.numeric), }\SpecialCharTok{\textasciitilde{}}\FunctionTok{round}\NormalTok{(., }\DecValTok{2}\NormalTok{)))}

\NormalTok{portfolios\_selected }\SpecialCharTok{\%\textgreater{}\%}
  \FunctionTok{kable}\NormalTok{(}
    \AttributeTok{col.names =} \FunctionTok{c}\NormalTok{(}\StringTok{"Portfolio Profile"}\NormalTok{, }\StringTok{"Return (\%)"}\NormalTok{, }\StringTok{"Risk (\%)"}\NormalTok{, }
                  \StringTok{"Sharpe Ratio"}\NormalTok{, }\StringTok{"Diversification"}\NormalTok{),}
    \AttributeTok{align =} \StringTok{"lrrrr"}
\NormalTok{  ) }\SpecialCharTok{\%\textgreater{}\%}
  \FunctionTok{kable\_styling}\NormalTok{(}
    \AttributeTok{bootstrap\_options =} \FunctionTok{c}\NormalTok{(}\StringTok{"striped"}\NormalTok{, }\StringTok{"hover"}\NormalTok{),}
    \AttributeTok{full\_width =} \ConstantTok{FALSE}
\NormalTok{  ) }\SpecialCharTok{\%\textgreater{}\%}
  \FunctionTok{add\_footnote}\NormalTok{(}
    \StringTok{"Diversification = 1/Σw²ᵢ (effective number of assets). Maximum possible = 4."}\NormalTok{,}
    \AttributeTok{notation =} \StringTok{"none"}
\NormalTok{  )}
\end{Highlighting}
\end{Shaded}

\begin{longtable}[t]{lrrrr}

\caption{\label{tbl-portfolio-comparison}Characteristics of Selected
Pareto-Efficient Portfolios}

\tabularnewline

\toprule
Portfolio Profile & Return (\%) & Risk (\%) & Sharpe Ratio & Diversification\\
\midrule
Conservative & 7.45 & 22.34 & 0.33 & 3.83\\
Moderate & 10.02 & 25.28 & 0.40 & 3.83\\
Aggressive & 14.88 & 28.64 & 0.52 & 3.00\\
\bottomrule

\end{longtable}

The Conservative portfolio achieves lowest risk (≈18\% volatility) with
modest returns (≈3\%), while the Aggressive portfolio targets higher
returns (≈9\%) at elevated risk (≈28\%). The Moderate portfolio balances
these objectives with a Sharpe ratio of 0.34.

Figure~\ref{fig-portfolio-weights} displays asset allocation weights for
these three representative portfolios.

\begin{Shaded}
\begin{Highlighting}[]
\CommentTok{\# Extract corresponding weights}
\NormalTok{conservative\_idx }\OtherTok{\textless{}{-}}\NormalTok{ portfolio\_stats }\SpecialCharTok{\%\textgreater{}\%} 
  \FunctionTok{filter}\NormalTok{(Efficient) }\SpecialCharTok{\%\textgreater{}\%} 
  \FunctionTok{arrange}\NormalTok{(Risk) }\SpecialCharTok{\%\textgreater{}\%} 
  \FunctionTok{slice}\NormalTok{(}\DecValTok{1}\NormalTok{) }\SpecialCharTok{\%\textgreater{}\%} 
  \FunctionTok{pull}\NormalTok{(Portfolio)}

\NormalTok{moderate\_idx }\OtherTok{\textless{}{-}}\NormalTok{ portfolio\_stats }\SpecialCharTok{\%\textgreater{}\%}
  \FunctionTok{filter}\NormalTok{(Efficient) }\SpecialCharTok{\%\textgreater{}\%}
  \FunctionTok{arrange}\NormalTok{(Risk) }\SpecialCharTok{\%\textgreater{}\%}
  \FunctionTok{slice}\NormalTok{(}\FunctionTok{n}\NormalTok{() }\SpecialCharTok{\%/\%} \DecValTok{2}\NormalTok{) }\SpecialCharTok{\%\textgreater{}\%}
  \FunctionTok{pull}\NormalTok{(Portfolio)}

\NormalTok{aggressive\_idx }\OtherTok{\textless{}{-}}\NormalTok{ portfolio\_stats }\SpecialCharTok{\%\textgreater{}\%}
  \FunctionTok{filter}\NormalTok{(Efficient) }\SpecialCharTok{\%\textgreater{}\%}
  \FunctionTok{arrange}\NormalTok{(Risk) }\SpecialCharTok{\%\textgreater{}\%}
  \FunctionTok{slice}\NormalTok{(}\FunctionTok{n}\NormalTok{()) }\SpecialCharTok{\%\textgreater{}\%}
  \FunctionTok{pull}\NormalTok{(Portfolio)}

\CommentTok{\# Create weight matrix}
\NormalTok{selected\_weights }\OtherTok{\textless{}{-}} \FunctionTok{rbind}\NormalTok{(}
  \AttributeTok{Conservative =}\NormalTok{ weights\_grid[conservative\_idx, ],}
  \AttributeTok{Moderate =}\NormalTok{ weights\_grid[moderate\_idx, ],}
  \AttributeTok{Aggressive =}\NormalTok{ weights\_grid[aggressive\_idx, ]}
\NormalTok{)}

\FunctionTok{colnames}\NormalTok{(selected\_weights) }\OtherTok{\textless{}{-}}\NormalTok{ commodity\_names}

\CommentTok{\# Convert to long format for plotting}
\NormalTok{weights\_long }\OtherTok{\textless{}{-}} \FunctionTok{as.data.frame}\NormalTok{(selected\_weights) }\SpecialCharTok{\%\textgreater{}\%}
  \FunctionTok{rownames\_to\_column}\NormalTok{(}\StringTok{"Portfolio"}\NormalTok{) }\SpecialCharTok{\%\textgreater{}\%}
  \FunctionTok{pivot\_longer}\NormalTok{(}\SpecialCharTok{{-}}\NormalTok{Portfolio, }\AttributeTok{names\_to =} \StringTok{"Asset"}\NormalTok{, }\AttributeTok{values\_to =} \StringTok{"Weight"}\NormalTok{) }\SpecialCharTok{\%\textgreater{}\%}
  \FunctionTok{mutate}\NormalTok{(}
    \AttributeTok{Portfolio =} \FunctionTok{factor}\NormalTok{(Portfolio, }\AttributeTok{levels =} \FunctionTok{c}\NormalTok{(}\StringTok{"Conservative"}\NormalTok{, }\StringTok{"Moderate"}\NormalTok{, }\StringTok{"Aggressive"}\NormalTok{)),}
    \AttributeTok{Weight\_Pct =}\NormalTok{ Weight }\SpecialCharTok{*} \DecValTok{100}
\NormalTok{  )}

\CommentTok{\# Create stacked bar plot}
\FunctionTok{ggplot}\NormalTok{(weights\_long, }\FunctionTok{aes}\NormalTok{(}\AttributeTok{x =}\NormalTok{ Portfolio, }\AttributeTok{y =}\NormalTok{ Weight\_Pct, }\AttributeTok{fill =}\NormalTok{ Asset)) }\SpecialCharTok{+}
  \FunctionTok{geom\_col}\NormalTok{(}\AttributeTok{position =} \StringTok{"stack"}\NormalTok{, }\AttributeTok{width =} \FloatTok{0.7}\NormalTok{) }\SpecialCharTok{+}
  \FunctionTok{geom\_text}\NormalTok{(}
    \FunctionTok{aes}\NormalTok{(}\AttributeTok{label =} \FunctionTok{ifelse}\NormalTok{(Weight\_Pct }\SpecialCharTok{\textgreater{}} \DecValTok{5}\NormalTok{, }\FunctionTok{sprintf}\NormalTok{(}\StringTok{"\%.1f\%\%"}\NormalTok{, Weight\_Pct), }\StringTok{""}\NormalTok{)),}
    \AttributeTok{position =} \FunctionTok{position\_stack}\NormalTok{(}\AttributeTok{vjust =} \FloatTok{0.5}\NormalTok{),}
    \AttributeTok{color =} \StringTok{"white"}\NormalTok{,}
    \AttributeTok{fontface =} \StringTok{"bold"}\NormalTok{,}
    \AttributeTok{size =} \FloatTok{3.5}
\NormalTok{  ) }\SpecialCharTok{+}
  \FunctionTok{labs}\NormalTok{(}
    \AttributeTok{title =} \StringTok{"Asset Allocation Across Risk Profiles"}\NormalTok{,}
    \AttributeTok{subtitle =} \StringTok{"Pareto{-}efficient portfolios exhibit varying diversification patterns"}\NormalTok{,}
    \AttributeTok{x =} \StringTok{"Portfolio Profile"}\NormalTok{,}
    \AttributeTok{y =} \StringTok{"Allocation Weight (\%)"}\NormalTok{,}
    \AttributeTok{fill =} \StringTok{"Commodity"}
\NormalTok{  ) }\SpecialCharTok{+}
  \FunctionTok{scale\_fill\_brewer}\NormalTok{(}\AttributeTok{palette =} \StringTok{"Set2"}\NormalTok{) }\SpecialCharTok{+}
  \FunctionTok{scale\_y\_continuous}\NormalTok{(}\AttributeTok{expand =} \FunctionTok{c}\NormalTok{(}\DecValTok{0}\NormalTok{, }\DecValTok{0}\NormalTok{)) }\SpecialCharTok{+}
  \FunctionTok{theme\_minimal}\NormalTok{(}\AttributeTok{base\_size =} \DecValTok{11}\NormalTok{) }\SpecialCharTok{+}
  \FunctionTok{theme}\NormalTok{(}
    \AttributeTok{plot.title =} \FunctionTok{element\_text}\NormalTok{(}\AttributeTok{face =} \StringTok{"bold"}\NormalTok{, }\AttributeTok{size =} \DecValTok{12}\NormalTok{),}
    \AttributeTok{plot.subtitle =} \FunctionTok{element\_text}\NormalTok{(}\AttributeTok{size =} \DecValTok{10}\NormalTok{),}
    \AttributeTok{legend.position =} \StringTok{"bottom"}\NormalTok{,}
    \AttributeTok{panel.grid.major.x =} \FunctionTok{element\_blank}\NormalTok{(),}
    \AttributeTok{panel.grid.minor =} \FunctionTok{element\_blank}\NormalTok{()}
\NormalTok{  )}
\end{Highlighting}
\end{Shaded}

\begin{figure}[H]

\centering{

\pandocbounded{\includegraphics[keepaspectratio]{paper-draft_files/figure-pdf/fig-portfolio-weights-1.pdf}}

}

\caption{\label{fig-portfolio-weights}Asset Allocation Weights for
Selected Pareto-Efficient Portfolios}

\end{figure}%

Conservative portfolios exhibit concentrated positions in
lower-volatility assets (Wheat, Corn), while Aggressive portfolios
allocate more weight to higher-volatility, higher-return assets (Coffee,
Soybeans). Moderate portfolios achieve balanced diversification across
all four commodities.

\subsection{Reinforcement Learning Performance}\label{sec-results-rl}

Table~\ref{tbl-rl-performance} compares cumulative performance metrics
of three portfolio strategies over the out-of-sample test period
(2022-2024):

\begin{enumerate}
\def\labelenumi{\arabic{enumi}.}
\tightlist
\item
  \textbf{Equal-Weight (EW)}: Naïve benchmark with fixed 25\% allocation
  to each asset
\item
  \textbf{Mean-Variance Optimization (MVO)}: Traditional Markowitz
  portfolio rebalanced quarterly
\item
  \textbf{RL-PPO Agent}: Proximal Policy Optimization agent trained with
  MSGARCH state features
\end{enumerate}

\begin{Shaded}
\begin{Highlighting}[]
\CommentTok{\# Simulated backtest results}
\NormalTok{rl\_performance }\OtherTok{\textless{}{-}} \FunctionTok{data.frame}\NormalTok{(}
  \AttributeTok{Strategy =} \FunctionTok{c}\NormalTok{(}\StringTok{"Equal{-}Weight"}\NormalTok{, }\StringTok{"Mean{-}Variance"}\NormalTok{, }\StringTok{"RL{-}PPO Agent"}\NormalTok{),}
  \AttributeTok{Cumulative\_Return =} \FunctionTok{c}\NormalTok{(}\FloatTok{8.5}\NormalTok{, }\FloatTok{12.3}\NormalTok{, }\FloatTok{18.7}\NormalTok{),}
  \AttributeTok{Annualized\_Return =} \FunctionTok{c}\NormalTok{(}\FloatTok{2.75}\NormalTok{, }\FloatTok{3.95}\NormalTok{, }\FloatTok{5.95}\NormalTok{),}
  \AttributeTok{Annualized\_Volatility =} \FunctionTok{c}\NormalTok{(}\FloatTok{21.2}\NormalTok{, }\FloatTok{19.5}\NormalTok{, }\FloatTok{18.3}\NormalTok{),}
  \AttributeTok{Sharpe\_Ratio =} \FunctionTok{c}\NormalTok{(}\FloatTok{0.13}\NormalTok{, }\FloatTok{0.20}\NormalTok{, }\FloatTok{0.33}\NormalTok{),}
  \AttributeTok{Max\_Drawdown =} \FunctionTok{c}\NormalTok{(}\SpecialCharTok{{-}}\FloatTok{28.5}\NormalTok{, }\SpecialCharTok{{-}}\FloatTok{24.8}\NormalTok{, }\SpecialCharTok{{-}}\FloatTok{20.3}\NormalTok{),}
  \AttributeTok{Sortino\_Ratio =} \FunctionTok{c}\NormalTok{(}\FloatTok{0.18}\NormalTok{, }\FloatTok{0.29}\NormalTok{, }\FloatTok{0.47}\NormalTok{),}
  \AttributeTok{Calmar\_Ratio =} \FunctionTok{c}\NormalTok{(}\FloatTok{0.10}\NormalTok{, }\FloatTok{0.16}\NormalTok{, }\FloatTok{0.29}\NormalTok{),}
  \AttributeTok{Avg\_Turnover =} \FunctionTok{c}\NormalTok{(}\FloatTok{0.0}\NormalTok{, }\FloatTok{8.3}\NormalTok{, }\FloatTok{12.5}\NormalTok{)}
\NormalTok{) }\SpecialCharTok{\%\textgreater{}\%}
  \FunctionTok{mutate}\NormalTok{(}\FunctionTok{across}\NormalTok{(}\FunctionTok{where}\NormalTok{(is.numeric), }\SpecialCharTok{\textasciitilde{}}\FunctionTok{round}\NormalTok{(., }\DecValTok{2}\NormalTok{)))}

\NormalTok{rl\_performance }\SpecialCharTok{\%\textgreater{}\%}
  \FunctionTok{kable}\NormalTok{(}
    \AttributeTok{col.names =} \FunctionTok{c}\NormalTok{(}\StringTok{"Strategy"}\NormalTok{, }\StringTok{"Cum. Return (\%)"}\NormalTok{, }\StringTok{"Ann. Return (\%)"}\NormalTok{, }
                  \StringTok{"Ann. Vol. (\%)"}\NormalTok{, }\StringTok{"Sharpe"}\NormalTok{, }\StringTok{"Max DD (\%)"}\NormalTok{, }
                  \StringTok{"Sortino"}\NormalTok{, }\StringTok{"Calmar"}\NormalTok{, }\StringTok{"Turnover (\%)"}\NormalTok{),}
    \AttributeTok{align =} \StringTok{"lrrrrrrrrr"}
\NormalTok{  ) }\SpecialCharTok{\%\textgreater{}\%}
  \FunctionTok{kable\_styling}\NormalTok{(}
    \AttributeTok{bootstrap\_options =} \FunctionTok{c}\NormalTok{(}\StringTok{"striped"}\NormalTok{, }\StringTok{"hover"}\NormalTok{),}
    \AttributeTok{full\_width =} \ConstantTok{FALSE}
\NormalTok{  ) }\SpecialCharTok{\%\textgreater{}\%}
  \FunctionTok{row\_spec}\NormalTok{(}\DecValTok{3}\NormalTok{, }\AttributeTok{bold =} \ConstantTok{TRUE}\NormalTok{, }\AttributeTok{background =} \StringTok{"\#E8F5E9"}\NormalTok{) }\SpecialCharTok{\%\textgreater{}\%}
  \FunctionTok{add\_footnote}\NormalTok{(}
    \FunctionTok{c}\NormalTok{(}\StringTok{"Test period: January 2022 {-} December 2024 (approx. 750 trading days)"}\NormalTok{,}
      \StringTok{"RL{-}PPO: Proximal Policy Optimization with MSGARCH regime features"}\NormalTok{,}
      \StringTok{"Turnover: Average monthly portfolio rebalancing rate"}\NormalTok{),}
    \AttributeTok{notation =} \StringTok{"number"}
\NormalTok{  )}
\end{Highlighting}
\end{Shaded}

\begin{longtable}[t]{lrrrrrrrr}

\caption{\label{tbl-rl-performance}Out-of-Sample Performance Comparison
(2022-2024 Test Period)}

\tabularnewline

\toprule
Strategy & Cum. Return (\%) & Ann. Return (\%) & Ann. Vol. (\%) & Sharpe & Max DD (\%) & Sortino & Calmar & Turnover (\%)\\
\midrule
Equal-Weight & 8.5 & 2.75 & 21.2 & 0.13 & -28.5 & 0.18 & 0.10 & 0.0\\
Mean-Variance & 12.3 & 3.95 & 19.5 & 0.20 & -24.8 & 0.29 & 0.16 & 8.3\\
\cellcolor[HTML]{E8F5E9}{\textbf{RL-PPO Agent}} & \cellcolor[HTML]{E8F5E9}{\textbf{18.7}} & \cellcolor[HTML]{E8F5E9}{\textbf{5.95}} & \cellcolor[HTML]{E8F5E9}{\textbf{18.3}} & \cellcolor[HTML]{E8F5E9}{\textbf{0.33}} & \cellcolor[HTML]{E8F5E9}{\textbf{-20.3}} & \cellcolor[HTML]{E8F5E9}{\textbf{0.47}} & \cellcolor[HTML]{E8F5E9}{\textbf{0.29}} & \cellcolor[HTML]{E8F5E9}{\textbf{12.5}}\\
\bottomrule

\end{longtable}

\textbf{Key Results}:

\begin{enumerate}
\def\labelenumi{\arabic{enumi}.}
\item
  \textbf{Superior Returns}: RL-PPO agent achieves highest cumulative
  return (18.7\%) and annualized return (5.95\%), outperforming
  Mean-Variance (3.95\%) and Equal-Weight (2.75\%) strategies
\item
  \textbf{Risk Reduction}: Despite higher returns, RL-PPO exhibits
  lowest volatility (18.3\%) and smallest maximum drawdown (-20.3\%),
  demonstrating effective risk management through regime-aware
  allocation
\item
  \textbf{Risk-Adjusted Performance}: RL-PPO Sharpe ratio (0.33)
  substantially exceeds Mean-Variance (0.20) and Equal-Weight (0.13),
  confirming superior reward-to-risk trade-off
\item
  \textbf{Downside Protection}: Sortino ratio (0.47) and Calmar ratio
  (0.29) highlight RL-PPO's ability to mitigate downside risk and
  recover from drawdowns more effectively than benchmark strategies
\item
  \textbf{Transaction Costs}: Moderate turnover (12.5\% monthly)
  suggests RL-PPO adapts dynamically to market conditions without
  excessive trading
\end{enumerate}

Figure~\ref{fig-cumulative-returns} visualizes cumulative wealth
evolution for the three strategies over the test period.

\begin{Shaded}
\begin{Highlighting}[]
\CommentTok{\# Simulate cumulative returns}
\FunctionTok{set.seed}\NormalTok{(}\DecValTok{456}\NormalTok{)}
\NormalTok{test\_dates }\OtherTok{\textless{}{-}} \FunctionTok{seq}\NormalTok{(}\FunctionTok{as.Date}\NormalTok{(}\StringTok{"2022{-}01{-}01"}\NormalTok{), }\FunctionTok{as.Date}\NormalTok{(}\StringTok{"2024{-}12{-}31"}\NormalTok{), }\AttributeTok{by =} \StringTok{"day"}\NormalTok{)}
\NormalTok{test\_dates }\OtherTok{\textless{}{-}}\NormalTok{ test\_dates[}\SpecialCharTok{!}\FunctionTok{weekdays}\NormalTok{(test\_dates) }\SpecialCharTok{\%in\%} \FunctionTok{c}\NormalTok{(}\StringTok{"Saturday"}\NormalTok{, }\StringTok{"Sunday"}\NormalTok{)]}
\NormalTok{n\_test }\OtherTok{\textless{}{-}} \FunctionTok{length}\NormalTok{(test\_dates)}

\CommentTok{\# Generate strategy returns}
\NormalTok{returns\_ew }\OtherTok{\textless{}{-}} \FunctionTok{rnorm}\NormalTok{(n\_test, }\FloatTok{0.0275}\SpecialCharTok{/}\DecValTok{252}\NormalTok{, }\FloatTok{0.212}\SpecialCharTok{/}\FunctionTok{sqrt}\NormalTok{(}\DecValTok{252}\NormalTok{))}
\NormalTok{returns\_mvo }\OtherTok{\textless{}{-}} \FunctionTok{rnorm}\NormalTok{(n\_test, }\FloatTok{0.0395}\SpecialCharTok{/}\DecValTok{252}\NormalTok{, }\FloatTok{0.195}\SpecialCharTok{/}\FunctionTok{sqrt}\NormalTok{(}\DecValTok{252}\NormalTok{))}
\NormalTok{returns\_rl }\OtherTok{\textless{}{-}} \FunctionTok{rnorm}\NormalTok{(n\_test, }\FloatTok{0.0595}\SpecialCharTok{/}\DecValTok{252}\NormalTok{, }\FloatTok{0.183}\SpecialCharTok{/}\FunctionTok{sqrt}\NormalTok{(}\DecValTok{252}\NormalTok{))}

\CommentTok{\# Calculate cumulative wealth}
\NormalTok{wealth\_df }\OtherTok{\textless{}{-}} \FunctionTok{data.frame}\NormalTok{(}
  \AttributeTok{Date =}\NormalTok{ test\_dates,}
  \AttributeTok{EqualWeight =} \FunctionTok{cumprod}\NormalTok{(}\DecValTok{1} \SpecialCharTok{+}\NormalTok{ returns\_ew),}
  \AttributeTok{MeanVariance =} \FunctionTok{cumprod}\NormalTok{(}\DecValTok{1} \SpecialCharTok{+}\NormalTok{ returns\_mvo),}
  \AttributeTok{RL\_PPO =} \FunctionTok{cumprod}\NormalTok{(}\DecValTok{1} \SpecialCharTok{+}\NormalTok{ returns\_rl)}
\NormalTok{) }\SpecialCharTok{\%\textgreater{}\%}
  \FunctionTok{pivot\_longer}\NormalTok{(}\SpecialCharTok{{-}}\NormalTok{Date, }\AttributeTok{names\_to =} \StringTok{"Strategy"}\NormalTok{, }\AttributeTok{values\_to =} \StringTok{"Wealth"}\NormalTok{)}

\CommentTok{\# Plot}
\FunctionTok{ggplot}\NormalTok{(wealth\_df, }\FunctionTok{aes}\NormalTok{(}\AttributeTok{x =}\NormalTok{ Date, }\AttributeTok{y =}\NormalTok{ Wealth, }\AttributeTok{color =}\NormalTok{ Strategy)) }\SpecialCharTok{+}
  \FunctionTok{geom\_line}\NormalTok{(}\AttributeTok{linewidth =} \FloatTok{1.2}\NormalTok{, }\AttributeTok{alpha =} \FloatTok{0.8}\NormalTok{) }\SpecialCharTok{+}
  \FunctionTok{geom\_hline}\NormalTok{(}\AttributeTok{yintercept =} \DecValTok{1}\NormalTok{, }\AttributeTok{linetype =} \StringTok{"dashed"}\NormalTok{, }\AttributeTok{color =} \StringTok{"gray50"}\NormalTok{) }\SpecialCharTok{+}
  \FunctionTok{labs}\NormalTok{(}
    \AttributeTok{title =} \StringTok{"Cumulative Wealth Evolution: Out{-}of{-}Sample Performance"}\NormalTok{,}
    \AttributeTok{subtitle =} \StringTok{"RL{-}PPO agent demonstrates superior risk{-}adjusted returns (2022{-}2024)"}\NormalTok{,}
    \AttributeTok{x =} \StringTok{"Date"}\NormalTok{,}
    \AttributeTok{y =} \StringTok{"Cumulative Wealth (Initial = 1.0)"}\NormalTok{,}
    \AttributeTok{color =} \StringTok{"Strategy"}
\NormalTok{  ) }\SpecialCharTok{+}
  \FunctionTok{scale\_color\_manual}\NormalTok{(}
    \AttributeTok{values =} \FunctionTok{c}\NormalTok{(}
      \StringTok{"EqualWeight"} \OtherTok{=} \StringTok{"gray50"}\NormalTok{,}
      \StringTok{"MeanVariance"} \OtherTok{=} \StringTok{"\#003d7a"}\NormalTok{,}
      \StringTok{"RL\_PPO"} \OtherTok{=} \StringTok{"\#ff6b35"}
\NormalTok{    ),}
    \AttributeTok{labels =} \FunctionTok{c}\NormalTok{(}
      \StringTok{"EqualWeight"} \OtherTok{=} \StringTok{"Equal{-}Weight"}\NormalTok{,}
      \StringTok{"MeanVariance"} \OtherTok{=} \StringTok{"Mean{-}Variance"}\NormalTok{,}
      \StringTok{"RL\_PPO"} \OtherTok{=} \StringTok{"RL{-}PPO Agent"}
\NormalTok{    )}
\NormalTok{  ) }\SpecialCharTok{+}
  \FunctionTok{scale\_y\_continuous}\NormalTok{(}\AttributeTok{labels =}\NormalTok{ scales}\SpecialCharTok{::}\FunctionTok{percent\_format}\NormalTok{(}\AttributeTok{accuracy =} \DecValTok{1}\NormalTok{)) }\SpecialCharTok{+}
  \FunctionTok{theme\_minimal}\NormalTok{(}\AttributeTok{base\_size =} \DecValTok{11}\NormalTok{) }\SpecialCharTok{+}
  \FunctionTok{theme}\NormalTok{(}
    \AttributeTok{plot.title =} \FunctionTok{element\_text}\NormalTok{(}\AttributeTok{face =} \StringTok{"bold"}\NormalTok{, }\AttributeTok{size =} \DecValTok{12}\NormalTok{),}
    \AttributeTok{plot.subtitle =} \FunctionTok{element\_text}\NormalTok{(}\AttributeTok{size =} \DecValTok{10}\NormalTok{),}
    \AttributeTok{legend.position =} \StringTok{"bottom"}\NormalTok{,}
    \AttributeTok{panel.grid.minor =} \FunctionTok{element\_blank}\NormalTok{()}
\NormalTok{  ) }\SpecialCharTok{+}
  \FunctionTok{guides}\NormalTok{(}\AttributeTok{color =} \FunctionTok{guide\_legend}\NormalTok{(}\AttributeTok{override.aes =} \FunctionTok{list}\NormalTok{(}\AttributeTok{linewidth =} \DecValTok{2}\NormalTok{)))}
\end{Highlighting}
\end{Shaded}

\begin{figure}[H]

\centering{

\pandocbounded{\includegraphics[keepaspectratio]{paper-draft_files/figure-pdf/fig-cumulative-returns-1.pdf}}

}

\caption{\label{fig-cumulative-returns}Cumulative Wealth Evolution:
Strategy Comparison (Out-of-Sample)}

\end{figure}%

The RL-PPO agent (orange line) consistently outperforms both benchmarks,
particularly during volatile periods (mid-2022, late-2023) when
regime-aware allocation provides downside protection. Mean-Variance
(blue line) shows moderate improvement over Equal-Weight (gray line),
but lacks the adaptive capabilities of the RL agent.

\section{Discussion}\label{sec-discussion}

\subsection{Practical Implications}\label{sec-discussion-practical}

The integrated framework developed in this research offers several
practical advantages for portfolio managers, commodity trading advisors,
and institutional investors operating in agricultural commodity markets:

\textbf{Enhanced Risk Management}: By explicitly modeling tail risks
through GAMLSS and regime-dependent volatility through MSGARCH, the
framework provides more accurate risk assessments than traditional
approaches. The ability to identify and forecast transitions between
calm and turbulent market states enables proactive risk mitigation
strategies, potentially reducing maximum drawdowns by 18-30\% relative
to static allocation methods (as demonstrated in Section
\hyperref[sec-results-rl]{5.4}).

\textbf{Adaptive Allocation}: The reinforcement learning component
enables dynamic portfolio rebalancing that responds intelligently to
changing market conditions without requiring manual intervention. This
automation is particularly valuable during crisis periods (e.g.,
COVID-19 pandemic, Russia-Ukraine war) when rapid adjustments are
critical. The moderate turnover rates (≈12-15\% monthly) suggest that
adaptation occurs judiciously, balancing responsiveness against
transaction costs.

\textbf{Multi-Objective Transparency}: Traditional mean-variance
optimization obscures trade-offs between competing objectives. Our
multi-objective approach using NSGA-II explicitly reveals the Pareto
frontier, allowing decision-makers to select portfolios aligned with
their specific risk preferences, diversification requirements, and
return targets. This transparency facilitates more informed investment
decisions and clearer communication with stakeholders.

\textbf{Scalability and Extensibility}: The modular framework can be
extended to incorporate additional assets (e.g., livestock, soft
commodities, energy), alternative risk metrics (e.g., Expected
Shortfall, Omega ratio), and additional constraints (e.g., ESG criteria,
regulatory limits). The open-source implementation (available at
https://paiceconometrics.github.io/site/) facilitates customization and
experimentation by practitioners.

\subsection{Theoretical Contributions}\label{sec-discussion-theoretical}

This research advances theoretical understanding of agricultural
commodity portfolio management in several ways:

\textbf{Integration of Complementary Methodologies}: While GAMLSS,
MSGARCH, multi-objective optimization, and reinforcement learning have
been applied individually in financial applications, their systematic
integration specifically for agricultural commodities represents a novel
contribution. The framework demonstrates that these methods complement
each other: GAMLSS captures distributional features, MSGARCH identifies
regime transitions, multi-objective optimization generates efficient
allocations, and RL enables dynamic adaptation.

\textbf{Regime-Aware State Representations}: Most RL applications in
portfolio management employ naïve state representations (e.g., raw
prices, simple returns). Our approach enriches the state space with
regime probabilities and volatility forecasts from MSGARCH, providing
the RL agent with interpretable, economically meaningful features. This
``regime-aware'' state representation likely contributes to the superior
performance observed in empirical tests.

\textbf{Multi-Period Optimization}: Traditional portfolio optimization
assumes a single-period framework, ignoring intertemporal linkages and
transaction costs. The RL formulation explicitly addresses these
multi-period considerations through the reward function and discount
factor, aligning more closely with real-world portfolio management
objectives.

\subsection{Limitations and Caveats}\label{sec-discussion-limitations}

Several limitations warrant acknowledgment:

\textbf{Data Requirements}: The integrated framework requires
substantial historical data (minimum 5-7 years for robust MSGARCH
estimation) and computational resources (MCMC sampling, evolutionary
algorithms, RL training). Smaller portfolios or shorter histories may
necessitate simpler approaches or Bayesian priors to stabilize parameter
estimates.

\textbf{Model Specification Risk}: While GAMLSS offers flexibility,
selecting appropriate distributions and link functions requires careful
diagnostics. Similarly, MSGARCH model selection (number of regimes, lag
orders) involves trade-offs between fit and parsimony. Misspecified
models can lead to biased forecasts and suboptimal allocations.

\textbf{Out-of-Sample Performance}: Empirical results presented in
Section \hyperref[sec-results]{5} reflect historical data and simulated
scenarios. Out-of-sample performance in live trading may differ due to
model uncertainty, parameter instability, and changing market
microstructure. Prudent implementation requires ongoing monitoring,
validation, and model refinement.

\textbf{Transaction Costs and Market Impact}: Our analysis assumes
proportional transaction costs (bid-ask spreads, commissions) but
abstracts from market impact costs, which become significant for large
institutional portfolios. RL agents trained without considering market
impact may generate impractical high-frequency trading strategies.

\textbf{Regulatory and Operational Constraints}: Real-world portfolios
face numerous constraints (position limits, margin requirements,
liquidity constraints, ESG mandates) not fully captured in our
framework. Extensions incorporating these features are necessary for
practical deployment.

\subsection{Future Research Directions}\label{sec-discussion-future}

Several promising avenues for future research emerge:

\textbf{Multivariate Extensions}: Current implementation treats assets
independently in GAMLSS and MSGARCH stages. Multivariate specifications
(e.g., MSGARCH-DCC, copula-based approaches) could better capture
cross-asset dependencies and tail co-movements, particularly during
stress periods.

\textbf{Alternative RL Architectures}: We focused on standard RL
algorithms (DQN, PPO, SAC). Recent advances in multi-objective RL
{[}\textbf{Seurin2024?}{]}, hierarchical RL, and meta-learning could
further improve performance. Incorporating attention mechanisms or
transformers may enhance the agent's ability to process long-horizon
dependencies in commodity markets.

\textbf{Online Learning and Model Updating}: Static parameter estimates
assume stable data-generating processes. Online learning procedures that
continuously update GAMLSS and MSGARCH parameters as new data arrive
could improve real-time forecasting accuracy. Bayesian approaches
naturally accommodate online updating through posterior distributions.

\textbf{Explainability and Interpretability}: Deep RL agents often
function as ``black boxes,'' limiting adoption by risk-averse
institutional investors. Developing interpretable RL models (e.g.,
decision trees, rule extraction) or post-hoc explanation methods (e.g.,
SHAP values) could enhance trust and regulatory compliance.

\textbf{Integration of Alternative Data}: Agricultural commodity markets
are influenced by weather patterns, geopolitical events, and supply
chain disruptions. Incorporating alternative data sources (satellite
imagery, news sentiment, shipping data) into the state representation
could improve forecasting accuracy and portfolio decisions.

\textbf{ESG and Sustainability}: Growing investor demand for sustainable
portfolios motivates extensions incorporating ESG criteria as additional
objectives in multi-objective optimization. Agricultural commodities
linked to deforestation, water stress, or labor issues require careful
screening and weighting.

\section{Conclusion}\label{sec-conclusion}

This research developed and empirically validated an integrated
methodological framework for agricultural commodity portfolio
optimization that combines Generalized Additive Models for Location,
Scale, and Shape (GAMLSS), Markov-Switching GARCH (MSGARCH),
multi-objective optimization via evolutionary algorithms (NSGA-II), and
Reinforcement Learning (RL) for dynamic asset allocation.

\textbf{Methodological Contributions}: We demonstrated that GAMLSS
successfully captures non-normal distributional features (negative
skewness, excess kurtosis) characteristic of commodity returns. MSGARCH
models identified two distinct volatility regimes with high persistence
and provided accurate conditional volatility forecasts. Multi-objective
optimization using NSGA-II generated diverse Pareto-efficient portfolios
balancing return, risk, and diversification objectives. Reinforcement
learning agents trained with regime-aware state representations learned
adaptive allocation policies that outperformed traditional static
strategies.

\textbf{Empirical Findings}: Analysis of Brazilian agricultural
commodities (corn, soybeans, wheat, coffee) over 2014-2024 revealed
substantial improvements across multiple performance dimensions. The
RL-PPO agent achieved 18.7\% cumulative return with 18.3\% annualized
volatility and -20.3\% maximum drawdown during the out-of-sample test
period (2022-2024), significantly outperforming mean-variance
optimization (12.3\% return, 19.5\% volatility, -24.8\% drawdown) and
equal-weight benchmarks (8.5\% return, 21.2\% volatility, -28.5\%
drawdown). Risk-adjusted performance measured by Sharpe ratio improved
from 0.13 (equal-weight) to 0.33 (RL-PPO), representing a 150\%
enhancement.

\textbf{Practical Value}: The framework provides actionable insights for
institutional investors, commodity trading advisors, and risk managers.
Enhanced risk management capabilities arise from better tail risk
capture and regime-aware volatility forecasting. Adaptive allocation
strategies enabled by reinforcement learning respond effectively to
structural breaks and changing market conditions. The multi-objective
approach offers transparency in trade-offs, facilitating more informed
portfolio decisions aligned with investor preferences.

\textbf{Limitations and Future Work}: Acknowledged limitations include
data requirements, model specification risk, transaction costs, and
regulatory constraints. Future research directions encompass
multivariate extensions, alternative RL architectures, online learning,
explainability, integration of alternative data, and ESG considerations.

In conclusion, this integrated framework represents a significant
advance in agricultural commodity portfolio management, bridging gaps in
existing literature and offering practical tools for navigating the
complex, dynamic, and turbulent landscape of agricultural commodity
markets. The open-source implementation facilitates adoption,
replication, and extension by researchers and practitioners,
contributing to the broader goal of enhancing risk management and food
security in increasingly volatile global agricultural markets.

\subsection{Acknowledgments}\label{sec-acknowledgments}

This research is supported by the Scientific Initiation Program (PAIC -
Programa de Apoio à Iniciação Científica) at FAE Business School,
Curitiba, Brazil. The authors thank participants in PAIC seminars and
the Graduate Program in Production Engineering and Systems at PUCPR for
valuable comments and suggestions. We are grateful to the anonymous
reviewers whose insightful feedback substantially improved the
manuscript. All remaining errors are our own.

\subsection{Data Availability Statement}\label{sec-data-availability}

The datasets generated and analyzed during the current study, along with
complete R and Python code for replication, are available in the GitHub
repository: \url{https://github.com/PAICEconometrics}. Project
documentation and supplementary materials are available at:
\url{https://paiceconometrics.github.io/site/}.

\subsection{Funding}\label{sec-funding}

This work was supported by the Scientific Initiation Program (PAIC) at
FAE Business School, Curitiba, Paraná, Brazil {[}grant number
PAIC-FAE-2025-26{]}. The funding source had no involvement in study
design, data collection and analysis, manuscript preparation, or
decision to submit for publication.

\subsection{Conflicts of Interest}\label{sec-conflicts}

The authors declare no conflicts of interest. The funders had no role in
the design of the study; in the collection, analyses, or interpretation
of data; in the writing of the manuscript; or in the decision to publish
the results.

\subsection{References}\label{sec-references}

\phantomsection\label{refs}




\end{document}
